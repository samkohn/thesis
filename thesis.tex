\documentclass{ucbthesis}

\usepackage{siunitx}
\sisetup{
    math-micro=\ensuremath{\mu},
    text-micro=\ensuremath{\mu}
}

\newcommand{\tc}{\ensuremath{T_c}}
\newcommand{\fold}[1]{\ensuremath{#1}-fold}

\begin{document}

\chapter{Neutrino Oscillations}

\chapter{The Daya Bay Reactor Antineutrino Experiment}

\chapter{Calibration}

\chapter{Reconstruction}

\chapter{Event selection}

The event selection extracts pairs of signals
in the data stream that have properties expected of
IBD interactions: a prompt positron annihilation
followed by a delayed neutron capture on hydrogen.
At each step in the selection, it is critical that
differences in cut efficiency between the ADs are both
minimized and quantified so that any observed near-far difference
can be justifiably attributed to oscillations.

Certain backgrounds, namely muons, PMT flashers, and accidental coincidences,
are frequent enough that their characterization, veto, and/or subtraction
are handled as part of the event selection process.
I will briefly describe these backgrounds inline
but leave detailed descriptions and studies
to the appropriate sections in Chapter \ref{ch:background}.

\section{Initial data preparation}

Before physics events are identified,
the data is cleaned in two ways.
First, cosmogenic muon events are identified
and a time window after each muon is vetoed
to allow for any spallation products or activated nuclei to decay
without contaminating the IBD signal (Section \ref{sec:muonveto}).
Second, occurrences of PMT light emission, known as ``flashers,''
are also vetoed and removed from the data stream (Section \ref{subsec:flashers}).

\section{Coincidence selection}

The first step in the event selection is
to group together signals that are close together in time
into ``coincidence groups.''
Each \SI{1}{\micro\second} triggered readout window
with reconstructed energy above \SI{1.5}{\mega\electronvolt}
is identified as an ``AD event''
and is a potential coincidence candidate.
Because of the nonzero length of the readout window,
AD events occuring closer together than \SI{1}{\micro\second}
are not necessarily separate physical events.
Consequently, during the coincidence grouping process,
the coincidence search window begins \SI{1}{\micro\second}
after the initial AD event.
Coincidence groups are constructed by repeating the following steps
until the data file is exhausted:

\begin{enumerate}
    \item Find the next AD event.
        This AD event will be the ``prompt'' event of the coincidence group.
    \item Find all subsequent AD events within the desired coincidence time \tc.
        If a muon event is encountered within \tc,
        veto the entire coincidence group starting with the prompt event.
        (This additional vetoed time is accounted for in the muon veto efficiency.)
    \item Group these events together with the prompt event
        to form the coincidence group.
    \item Skip to the next AD event that is not part of the coincidence group.
\end{enumerate}

The total number of AD events in the group
is the multiplicity of the group.
Because of the initial \SI{1}{\micro\second} gap,
the actual time interval covered by any given coincidence window is
$\tc - \SI{1}{\micro\second}$.
This analysis uses a coincidence search window of $\tc = \SI{1.5}{\milli\second}$.

A coincidence group with multiplicity $n$ is also referred to
as an \fold{n} coincidence.
If a prompt event has no subsequent AD events within \tc, it is
still a valid group, and is referred to as a \fold{1} coincidence.
Note that \fold{1} coincidences are somewhat but not strictly isolated
from other AD events.
Certainly there are no other AD events
within \tc{} \textit{after} the prompt event,
but there may be a \textit{preceding} AD event within \tc{}
if that event is part of a coincidence window
which ends before the prompt event in question.

Once the coincidence groups have been constructed,
the set of \fold{2} coincidences can be identified as
the preliminary set of IBD candidates,
albeit with background still present.
The \fold{1} coincidences can be identified as a subset
of the uncorrelated events, mostly radioactive decays,
that are also present in the data stream.
However, not all uncorrelated events end up in \fold{1} coincidences.
Sometimes an uncorrelated event will occur in close proximity to
a true IBD prompt-delayed pair, creating a \fold{3} coincidence.
These high-multiplicity coincidence groups are vetoed
with a small loss of efficiency (Section \ref{subsec:acc}).
More concerning is when two uncorrelated events
randomly occur in close proximity to each other,
creating a \fold{2} coincidence group that passes the high-multiplicity veto.
These so-called ``accidental'' coincidences
constitute the largest background within the set of \fold{2} coincidences.
The distance, time and energy cuts described below
are all motivated in large part by the need to reduce the accidental background.

\section{Distance and time cuts}

The distance and time distributions between prompt and delayed AD events
are different depending on the physical process producing those AD event pairs.
For example, the neutron produced during an IBD interaction
scatters within the liquid scintillator until being captured
by a hydrogen nucleus,
traveling a characteristic distance over a characteristic time.
On the other hand, two uncorrelated events have, by definition,
no particular connection between their physical locations
or their timings.

In practice, the characteristic distance for a neutron capture on hydrogen
is approximately \SI{200}{\milli\meter},
and the characteristic time delay is \SI{200}{\micro\second}.
For accidental coincidences, the characteristic distance is
the length scale of the AD, approximately \SI{3000}{\milli\meter},
and the time delay has a flat probability distribution
on the time scale used for the coincidence window ($\tc = \SI{1.5}{\milli\second}$).

The correlations for IBD events go further:
neutrons which capture sooner also tend to capture closer to the prompt event.
In other words, the distance and time distributions are correlated
on an event-by-event basis.
A downside to this fact is that selecting events first using a distance cut
and then using a time cut, or vice versa,
would lead to partially correlated uncertainties on the cut efficiencies
that would be difficult to characterize.
However, this correlation can be exploited to create a single cut
combining the distance and time that is straightforward to work with.

The exact combination of parameters for the cut was chosen based on observations
of the coincidence distance vs. coincidence time distributions.
This cut is known as the D-T cut and has the form

\begin{equation}
    \Delta r + v_0 \Delta t < \SI{800}{\milli\meter}
\end{equation}

where $v_0 = \frac{\SI{1000}{\milli\meter}}{\SI{600}{\micro\second}}$.
Applying this cut rejects the vast majority of accidental events
at a loss of approximately \SI{30}{\percent} of real IBDs.
The efficiency is measured after subtracting the accidental background
by comparing the number of events that pass the energy cuts and the D-T cut
with the number of events that pass the energy cuts
and a significantly relaxed D-T cut of \SI{3000}{\milli\meter}.
This latter quantity is assumed to be the total number of IBDs
since negligibly few IBDs have a D-T value anywhere close to \SI{3000}{\milli\meter}.

The AD-uncorrelated uncertainty for the efficiency
is determined from the data by examining the variation
in measured efficiency between the 4 near-hall ADs.
The far-hall ADs are excluded because
their statistical uncertainties are much larger than the
near-hall AD variation.
The AD-uncorrelated uncertainty of the D-T cut efficiency
is the half-range of the near-hall efficiencies: \num{0.0016} (absolute),
or approximately \SI{0.23}{\percent} (relative).

\section{Energy cuts}


\chapter{Background}
\label{ch:background}

\section{Muon veto}
\label{sec:muonveto}

\section{Uncorrelated background}

\subsection{PMT light emission / flashers}
\label{subsec:flashers}

\subsection{Accidental coincidences}
\label{subsec:acc}

\chapter{Measurement of $\theta_{13}$}

\chapter{Conclusions}

\end{document}
