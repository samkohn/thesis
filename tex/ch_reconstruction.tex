\chapter{Reconstruction}
\label{ch:reconstruction}

Each event within an AD is assigned a reconstructed position and energy
that take into account the pattern of PMT hits, the total light emitted,
scintillator and mineral oil optical characteristics,
spatial nonuniformity, and scintillator and electronics nonlinearity.
The reconstructed positions are used to compute the distance-time cut
to select IBD events as described in \cref{sec:DT_cut},
and are also used as inputs to the energy reconstruction
to help correct for nonuniformities in the ADs' light collection
as a function of position.
The reconstructed energy is a critical input to the \thetaot{} analysis
due to the heavy reliance on energy cuts to select IBDs and reject background.
The relative performance of ADs in reconstructing energy
is particularly important, since inconsistencies in energy reconstruction
could lead to unaccounted-for differences in efficiency,
which would bias the measurement of \thetaot.
The position reconstruction procedure will be detailed in \cref{sec:reco_position}.
Energy reconstruction, including the energy scale, determination of event energy,
and the nonuniformity and nonlinearity corrections,
will be described in \cref{sec:reco_energy}.

\section{Position}
\label{sec:reco_position}

Daya Bay uses two independent position reconstruction algorithms,
both of which rely on the pattern of charge measurements across all PMTs in an AD.
The reconstruction used in this \thetaot{} analysis is known as ``AdSimpleNL;''
the name was inherited from a predecessor algorithm which was rather simple,
and, as will be described in \cref{sec:reco_energy}, the ``NL'' refers to the
scintillator nonlinearity correction to the energy,
which is applied as the last step of the reconstruction.
The other reconstruction is called ``AdScaled.''
Briefly, a center-of-charge position is computed,
averaging over each PMT position weighted by the charge on the PMT,
and a parametrized correction derived from simulation
is applied to determine the reconstructed position.

AdSimpleNL is (again, counterintuitively), a more complex algorithm.
Each event's PMT charge pattern is compared to a library of \num{9600} templates
generated using a Monte Carlo simulation.
Each template represents one position on an $(r, \phi, z)$ grid
with \num{20} $r$ positions, \num{24} $\phi$ positions,
and \num{20} $z$ positions.
A $\chi^2$ is constructed to quantify the agreement between the measured charge pattern
and each of the templates:

\begin{equation}
    \chi^2(\textbf{r}_{\text{rec}}) = \sum_{i=1}^{192} -2\ln\frac{
        \text{Poisson}(N_i^{\text{obs}} \vert N_i^{\text{template}}(\textbf{r}_{\text{rec}}))
    }
    {
        \text{Poisson}(N_i^{\text{obs}} \vert N_i^{\text{obs}})
    },
\end{equation}
where $i$ indexes over PMTs, $j$ indexes over the \num{9600} charge templates,
$N_i^{\text{obs}}$ is the number of photoelectrons observed in PMT $i$,
$N_{ij}^{\text{template}}$ is the prediction of tempate $j$ for PMT $i$,
and $\text{Poisson}(n\vert\mu)$ is the Poisson probability
to observe $n$ counts given an expected value of $\mu$.
The $\chi^2$ function is extended to all values of $\textbf{r}_{\text{rec}}$
within an AD using interpolation between the lattice points used in the simulation.
The value of $\textbf{r}_{\text{rec}}$ which minimizes $\chi^2$
is used as the reconstructed position.

\section{Energy}
\label{sec:reco_energy}

The reconstructed energy for an event is built up from the previously-described
calibration values and the observed signals in each PMT,
and then corrected for AD nonuniformity and nonlinearity.

