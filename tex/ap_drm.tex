\chapter{Conversion between true and reconstructed energy}
\label{ap:drm}

To predict the far ADs' \nuebar{} spectra based on
an observation at a near AD,
the observed prompt event energy must be converted
to a true \nuebar{} energy
so that the oscillation probability can be computed.
After the projection, the predicted true energy spectrum
must be converted back into a reconstructed spectrum
to allow for a comparison to the far AD observations.
The relation between true \nuebar{} energy
and reconstructed prompt energy
is known as the detector response.
The detector response for Daya Bay
depended primarily on the absolute energy scale (\cref{subsec:abs_energyscale})
and energy resolution (\cref{subsec:resolution}).
Other effects such as positron energy loss within the acrylic vessels
and partially-escaped $\gamma$-rays were also relevant.

The traditional formalism for representing and applying
the detector response to the model prediction will be presented
in \cref{sec:drm_traditional},
followed by a description in \cref{sec:drm_lbnl}
of the formalism used in this analysis (\cref{sec:prediction}).
A brief discussion of the relative merits of each method
will be given in \cref{sec:drm_comparison}.
In both formalisms, the true and reconstructed energy distributions
of an event sample are represented by histograms,
which themselves are represented by column vectors,
with each component corresponding to a bin of the histogram.
The binning of the true and reconstructed histograms
need not be the same.

\section{Matrix formalism}
\label{sec:drm_traditional}

The traditional detector response matrix formalism
was \emph{not} used in this analysis.
It is presented here to motivate the use of the alternate vector-based formalism
described below.

A given incident \nuebar{} with true energy $E_i$ falling in true energy bin $i$
participating in an IBD interaction at a Daya Bay AD
has a certain probability $M^{(b)}_i$ for being observed
with a prompt reconstructed energy $E^{(b)}$ falling in reconstructed energy bin $b$:
\begin{equation}\label{eq:drm_trad_prob}
    M^{(b)}_i \equiv P(E_i \text{ observed as } E^{(b)})
\end{equation}
The total number of events observed in reconstructed bin $b$
can be computed based on the probability for each true energy bin
to be observed in bin $b$, which can be represented as a matrix equation:
\begin{align}\label{eq:drm_trad_matrix}
    \begin{split}
        S^{(b)}_R &= \sum_i M^{(b)}_i S_{\nu,i} \\
                  &\Leftrightarrow \\
        S_R &= \mathbf{M} S_\nu,
    \end{split}
\end{align}
where $S_R$ is the spectrum (histogram) of observed reconstructed energies
and $S_\nu$ is the spectrum of true incident \nuebar{} energies.
To obtain $S_\nu$, the inverse of $\mathbf{M}$ can be used if it exists:
\begin{equation}
    S_\nu = \mathbf{M}^{-1} S_R.
\end{equation}

The matrix $\mathbf{M}$ is not necessarily square,
and even if square, there is no requirement that
the bins of $S_R$ align with those of $S_\nu$.
In general there is no guarantee of an inverse existing for square $\mathbf{M}$,
or even a left or right inverse for non-square $\mathbf{M}$.
For typical detector response matrices,
$\mathbf{M}$ has the form of a smeared diagonal matrix
(such as that in \cref{fig:drm}),
where the reconstructed energy is distributed over bins near the true energy
due to a finite energy resolution.
Intuitively, this smearing causes a loss of information.
Thus it should not be surprising that a typical detector response matrix
has either no inverse or a numerically-unstable (and therefore meaningless) inverse.
Detector response matrices do, however, have the property
that for each true energy bin $i$,
\begin{equation}\label{eq:drm_trad_unitarity}
    \sum_b M^{(b)}_i = 1,
\end{equation}
in order to preserve probability.
Intuitively, each incident \nuebar{} must end up \emph{somewhere}.

Supposing that an inverse can be found,
the Daya Bay near-to-far prediction process can proceed.
The extracted true spectrum at a near AD $S_\nu^{\text{near}}$
is converted to a true spectrum at a far AD
based on the respective AD-reactor distances, detector and reactor livetimes,
and the oscillation probability, here represented all as
the transformation $g$:
\begin{equation}\label{eq:drm_trad_transform}
    S_\nu^{\text{far}} = g(S_\nu^{\text{near}}) = g(\mathbf{M}^{-1}S_R^{\text{near}}).
\end{equation}
A precise definition of $g$ can be derived based on the formulas in
\cref{subsec:flux_fraction,subsec:extrapolation}.
To obtain the final prediction to compare
to the observed reconstructed spectrum at a far AD,
the detector response for the far AD should be applied.
In general, there is no requirement that two ADs
have the same detector response matrix $\mathbf{M}$.
However, the Daya Bay ADs have been constrained to have
identical responses to the sub-percent level.
To a good approximation, then, the same matrix can be used:
\begin{equation}\label{eq:drm_trad_final}
    S_R^{\text{far}} = \mathbf{M} \cdot g(\mathbf{M}^{-1}S_R^{\text{near}}).
\end{equation}

This equation hints at one of the advantages of
performing the relative near-to-far measurement:
if $g$ consisted only of a pure scalar multiplication ($g(S) = cS$),
then the response matrix and its inverse could be grouped together
and canceled.
In practice, $g$ deviates from pure scalar multiplication
only on the order of the disappearance probability
(maximum \SI{\sim8}{\percent}),
which depends on the true \nuebar{} energy and thus on the component of
$S_\nu^{\text{near}}$.
Thus the final result is only slightly different from
the scenario of perfect cancellation.
This near-insensitivity to the detector response is also present
in the so-called vector formalism
actually used in the oscillation analysis presented in this thesis.

\section{Vector formalism}
\label{sec:drm_lbnl}

For a given event observed in reconstructed energy bin $b$,
the probability it was caused by an \nuebar{} with true energy $E_i$
falling in bin $i$ is $f_i^{(b)}$:
\begin{equation}\label{eq:drm_lbnl_prob}
    f_i^{(b)} \equiv P(E^{(b)} \text{ caused by } E_i).
\end{equation}
The true energy spectrum of incident \nuebar{}'s responsible for
events in reconstructed bin $b$ can then be written as
\begin{equation}\label{eq:drm_lbnl_spec}
    S_{\nu,i}^{(b)} = f_i^{(b)} S_R^{(b)},
\end{equation}
where $S_{\nu,i}^{(b)}$ can be assembled into a vector $S_\nu^{(b)}$,
and $S_R^{(b)}$ is the number of events in reconstructed bin $b$ (a scalar).
A separate vector is needed for each bin of reconstructed energy.

The values of $f_i^{(b)}$ can be assembled into a matrix
similar to $\mathbf{M}$ in the matrix formalism.
However, this matrix never acts on a vector using matrix multiplication,
i.e.\ as a linear transformation,
thus only the componentwise notation will be used
to emphasize this distinction.
Nevertheless, this new object does obey a normalization rule
in order to conserve probability:
\begin{equation}\label{eq:drm_lbnl_unitarity}
    \sum_i f_i^{(b)} = 1.
\end{equation}
Intuitively, each observed event in reconstructed energy bin $b$
must have come from \emph{somewhere}.
Note that this sum is over the true energy dimension,
whereas the corresponding sum in \cref{eq:drm_trad_unitarity}
for the matrix formalism is over the reconstructed energy dimension.

At a far AD, the expected true energy spectrum of \nuebar{}'s
with reconstructed prompt energy in bin $b$
is different from the near AD true spectrum
due to different AD-reactor distances, different livetimes,
and neutrino oscillations.
In principle, they could also differ because of
a different detector response between the near and far ADs,
but the Daya Bay ADs have been demonstrated to have
nearly-identical detector responses to the sub-percent level.
The relationship between the near and far true energy spectra
can be represented by a transformation $g$ which acts on the near AD spectrum:
\begin{equation}\label{eq:drm_lbnl_transform}
    S_\nu^{(b),\,\text{far}} = g(S_\nu^{(b),\,\text{near}})
    = g((f_0^{(b)}S_R^{(b)}, f_1^{(b)}S_R^{(b)}, \ldots, f_n^{(b)}S_R^{(b)})).
\end{equation}
A precise definition of $g$ can be derived using the formulas in
\cref{subsec:flux_fraction,subsec:extrapolation}.
For this analysis, $g$ does not mix different components together,
and can be written
\begin{equation}\label{eq:drm_lbnl_g_components}
    g((x_0, x_1, \ldots, x_n)) = (g_0(x_0), g_1(x_1), \ldots, g_n(x_n)).
\end{equation}

Each component of $S_\nu^{(b),\,\text{far}}$ (say, component $i$) represents
a prediction of
the number of \nuebar{} events at the far AD with true energy in bin $i$
which were observed in reconstructed energy bin $b$.
Thus the total predicted number of events in reconstructed energy bin $b$ is
\begin{equation}\label{eq:drm_lbnl_final}
    S_R^{(b),\,\text{far}} = \sum_i S_{\nu,i}^{(b),\,\text{far}}
    = \sum_i g_i(f_i^{(b)}S_R^{(b),\,\text{near}}).
\end{equation}
This formula possesses a similar property to the corresponding formula
in the traditional formalism:
if all $g_i$'s are just multiplication by the same scalar ($g_i(x) = cx$),
then that scalar and $S_R^{(b),\,\text{near}}$
can be factored out of the summation,
and the resulting sum is by construction
equal to unity by \cref{eq:drm_lbnl_unitarity}.
Thus the result would be independent of the detector response.
Since for the best-fit \thetaot{}, the $g_i$'s are only different
by at most \SI{\sim8}{\percent},
it can be concluded that the specific detector response values used
have only a small impact on the predicted spectrum.

\section{Discussion}
\label{sec:drm_comparison}

The primary advantage of the vector formalism
is that it does not require the ability to invert the detector response matrix,
which is in practice usually intractable or not well-defined.
However, there is a hidden dependence in the vector formalism
which is not present in the matrix formalism.
The values for $f_i^{(b)}$ depend on knowledge of
the true incident \nuebar{} spectrum,
while $\mathbf{M}$ can be computed without such knowledge.
This can be seen by viewing \cref{eq:drm_trad_prob,eq:drm_lbnl_prob}
through the lens of a Monte Carlo simulation,
which is how the detector response was computed in this analysis.
In the former, a fixed number of IBD interactions can be simulated
in true \nuebar{} energy bin $i$ to determine the probability
that such an event would be observed in reconstructed energy bin $b$.
Thus the values of $M_i^{(b)}$ can be determined for all bins $b$
by only simulating events with true energy in bin $i$.
This process can be repeated independently for each true energy bin.
For the latter (vector formalism) equation,
the simulated events must have a true \nuebar{} energy distribution
that matches the actual (real-life) \nuebar{} energy spectrum
so that each reconstructed energy bin is populated
by events from a realistic distribution of true energies.
In particular, to determine even a single value of $f_i^{(b)}$,
simulations must be performed over all the true energy bins
in order to properly normalize the fraction of events
which originated from true energy bin $i$.
The dependence of $f_i^{(b)}$ on the overall true \nuebar{} spectrum
is acceptable in this case
due to the weak dependence of the final prediction (\cref{eq:drm_lbnl_final})
on $f_i^{(b)}$.

When making an absolute rather than relative measurement,
e.g. to extract a measurement of the true \nuebar{} spectrum on its own
\cite{reactorflux2017},
it may be necessary to use the matrix formalism.
So-called regularization techniques have been developed
to solve this type of linear inverse problem.
For this analysis, since the final result
had a weak dependence on the detector response,
the vector formalism was an adequate replacement.
