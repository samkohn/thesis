\chapter{Calibration}
\label{ch:calibration}

The digital readouts from the TDC and ADCs representing PMT signals
are converted into time and charge through the calibration process.

\begin{figure}
    \missingfigure{Test figure}
    \caption{Test image}
    \label{fig:test}
\end{figure}

\section{Calibration system}
\label{sec:calib_system}

The calibration system for the Daya Bay ADs
allows for automated deployment of a variety of calibration sources.
Each AD is outfitted with three automated calibration units (ACUs)
which can position a calibration source at arbitrary locations
along the vertical axis extending underneath the each ACU.
ACU-A deploys sources along the central axis of the AD at $r=0$,
ACU-B probes the edge of the GdLS region just inside the IAV at $r=\SI{1350}{\mm}$,
and ACU-C can access the LS region between the IAV and OAV,
near the periphery of the AD at $r=\SI{1772.5}{\mm}$.
The layout of the ACUs is shown in \cref{fig:ad_cutaway},
and the construction and function of the ACUs
is described in detail in \cite{calib2014}.
Sources are deployed weekly during special calibration runs.

Each ACU contains three radioactive sources and one LED
which can be deployed into the AD during detector calibration,
as listed in \cref{tab:calibsources}.
The \isotope[60]{Co} and \isotope[241]{Am}-\isotope[13]{C} sources
are housed in the same fixture and so are always deployed together.
The ACUs can deploy any of the sources to a given vertical position
with a precision of \SI{5}{\mm},
the same precision as the positions of the PMTs.

\begin{table}[ht]
    \centering
    \begin{tabular}[t]{llll}
        \hline
        Source & Energy & Radiation & Rate \\
        \hline
        \isotope[60]{Co} & \SIlist{1.173;1.323}{\MeV} & $\gamma$-rays & \SI{100}{\Hz} \\
        \isotope[241]{Am}-\isotope[13]{C} & \SIrange{3}{6}{\MeV} & neutron &
            \SI{0.7}{\Hz} \\
        \isotope[68]{Ge} & $2\times\SI{0.511}{\MeV}$ & positrons & \SI{10}{\Hz} \\
        LED & $\lambda < \SI{435}{\nm}$ & UV photons & tbd \\
        \hline
    \end{tabular}
    \caption{The 4 calibration sources used in each ACU (\cite{calib2014,amc2015})
    \todo[inline]{LED pulse rate}}
    \label{tab:calibsources}
\end{table}

\section{Gain calibration}
\label{sec:gain}

\section{Time calibration}
\label{sec:time_calib}

\section{Channel quality}
\label{sec:channel_quality}

\section{Misc}

The LED has a maximum wavelength of \SI{435}{\nm}
and is used to calibrate the PMT gain and timing.
The amplitude of the LED can be varied to measure the PMT response
for different light levels.
The LED pulse width is $\sim\SI{10}{\ns}$ to allow for
precise measurements of differing response times among the PMTs.
The specifics of the time and gain calibration are described below.

