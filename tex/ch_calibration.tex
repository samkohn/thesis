\chapter{Calibration}
\label{ch:calibration}

The digital readouts from the TDC and ADCs representing PMT signals
must be converted into physically-relevant quantities.
The TDC values represent hit times of PMTs,
and the ADCs represent the charge collected by the PMTs,
which ultimately must be traced back to the energy deposited
in the liquid scintillator of the ADs.
The conversion factors for these quantities are determined
through the calibration process.

The calibration system for the Daya Bay ADs
allows for automated deployment of a variety of calibration sources.
Each AD is outfitted with three automated calibration units (ACUs)
which can position a calibration source at arbitrary locations
along the vertical axis extending underneath the each ACU.
ACU-A deploys sources along the central axis of the AD at $r=0$,
ACU-B probes the edge of the GdLS region just inside the IAV at $r=\SI{1350}{\mm}$,
and ACU-C can access the LS region between the IAV and OAV,
near the periphery of the AD at $r=\SI{1772.5}{\mm}$.
The layout of the ACUs is shown in \cref{fig:ad_cutaway},
and the construction and function of the ACUs
is described in detail in \cite{calib2014}.
Sources are deployed weekly during special calibration runs.

Each ACU contains three radioactive sources and one LED
which can be deployed into the AD during detector calibration,
as listed in \cref{tab:calibsources}.
The \isotope[60]{Co} and \isotope[241]{Am}-\isotope[13]{C} sources
are housed in the same fixture and so are always deployed together.
The ACUs can deploy any of the sources to a given vertical position
with a precision of \SI{5}{\mm},
the same precision as the positions of the PMTs.

\begin{table}[ht]
    \centering
    \begin{tabular}[t]{llll}
        \hline
        Source & Energy & Radiation & Rate \\
        \hline
        \isotope[60]{Co} & \SIlist{1.173;1.333}{\MeV} & $\gamma$-rays & \SI{100}{\Hz} \\
        \isotope[241]{Am}-\isotope[13]{C} & \SIrange{3}{6}{\MeV} & neutron &
            \SI{0.7}{\Hz} \\
        \isotope[68]{Ge} & $2\times\SI{0.511}{\MeV}$ & positrons & \SI{10}{\Hz} \\
        LED & $\lambda < \SI{435}{\nm}$ & UV photons & tbd \\
        \hline
    \end{tabular}
    \caption{The 4 calibration sources used in each ACU (\cite{calib2014,amc2015})
    \todo[inline]{LED pulse rate}}
    \label{tab:calibsources}
\end{table}

\section{Gain calibration}
\label{sec:gain}

The gain of a PMT channel is the degree of amplification
of the photon signal,
measured in ADC counts per photoelectron (\si{\adc\per\pe}),
and depends on the individual PMT, the input voltage,
and environmental factors such as the temperature
and the ambient magnetic field.
Although the PMTs are shielded from ambient fields,
and the temperature within each AD is relatively (but not entirely) stable,
the gain for individual PMTs does drift over time.
To properly account for these changes,
the PMT gain is measured continuously during data-taking
through a process called ``rolling gain.''

Rolling gain is possible because of PMT dark noise,
which consists almost exclusively of single photoelectron (SPE) signals.
The readout window buffer for each trigger actually extends $\sim\SI{320}{\ns}$
before the trigger criterion is met,
so that every triggered readout contains a few hundred \si{\ns}
of data where there were no physics events in the AD, known as the noise window.
Dark noise recorded during the noise window is the most reliable source
of SPE signals.
For each PMT, the ADC values of signals obtained during the noise window
are accumulated and fit to obtain the mean ADC counts per SPE,
which is then interpreted as the gain for that PMT channel.
The model for the fit is at its core a convolution of
a Poisson distribution counting the number of PEs
with a Gaussian distribution modeling the resolution of
the amplification and digitization process \cite{ngd2016}:

\begin{equation}
    S_i(Q) = \sum_{n=1}^2 \frac{\mu^n e^{-\mu}}{n!}
    \frac{1}{\sigma_{SPE}\sqrt{2n\pi}}
    \exp
    \left(
        -\frac{(Q-n\overline{Q}^{SPE})^2}{2n\sigma^2_{SPE}}
    \right),
\end{equation}
where $i$ is the PMT index,
$\mu$ is the mean number of dark noise PEs per noise window for PMT $i$,
$\overline{Q}^{SPE}$ is the mean \si{\adc\per\pe} (i.e.\ the gain),
and $\sigma_{SPE}$ is the resolution of the PMT-ADC system.
$n$ is the actual number of PEs measured in a given noise window.
The sum nominally runs from $0$ to $\infty$ but
excludes $n=0$ because in that case there is no readout signal,
and is truncated at $n=2$ because there are negligibly few dark noise events
with more than \SI{1}{\pe}.
For each PMT $i$, the values $\mu,\overline{Q}^{SPE}\text{, and }\sigma_{SPE}$ are fit
to the dark noise distribution.
The actual fit must account for the ADC pedestal (baseline),
which is simply the value of the ADC output when there is no signal on the PMT.
In order for the fit to be meaningful, dark noise hits must be accumulated
for approximately six hours.
The rolling gain is therefore sensitive to almost any conceivable
environmental change that could substantially change the gain of any PMT.
The average gain for the PMTs in each AD (measured in \si{\adc\per\pe})
over time is shown in \cref{fig:gain}.
The gain is independently cross-checked during the weekly calibration runs
using the low-intensity LED to generate SPE samples.

\begin{figure}
    \missingfigure{Gains plot}
    \caption{PMT gains over time, as measured by the rolling gain.}
    \label{fig:gain}
\end{figure}

\section{Light yield calibration}
\label{sec:light_yield_calib}

The light yield characterizes the average
number of photoelectrons (PEs) observed by PMTs
per energy deposited in the liquid scintillator
by a physics event.
It is measured both by deploying the \isotope[60]{Co} calibration source
during weekly calibrations
and using a rolling method based on muon-induced spallation neutron
captures on Gd (nGd), which accumulates sufficient statistics
approximately once per day \todo{Spn calibration statistics}.

Spallation neutrons are produced by muon interactions in the rock,
water, and steel surrounding the ADs.
When these neutrons penetrate into the GdLS,
they will capture on a Gd nucleus, leading to the emission of $\gamma$-rays
with energy totalling either \SI{7.95}{\MeV} or \SI{8.54}{\MeV},
depending on the specific isotope of Gd.
Spallation neutrons are identified by searching for events
with energy between \SIlist{6;12}{\MeV} shortly after muon signals.
A background dataset is obtained with an offset time window
and subtracted from the energy distribution of the spallation neutron dataset.
The peaks from the two different isotopes overlap,
so the distribution is then fit with a double Crystal Ball function \cite{cbfunction}.
The location of the lower peak is then defined to represent \SI{7.95}{\MeV}.
\Cref{fig:lightyield} shows the light yield over time,
as measured by the spallation neutron captures.

\begin{figure}
    \missingfigure{Light yield plot}
    \caption{Light yield over time, as measured by the spallation neutron method.}
    \label{fig:lightyield}
\end{figure}

\section{Time calibration}
\label{sec:time_calib}

The times returned by the TDC for each AD must be corrected
to account for differences in cable length as well as
manufacturing differences that could affect the PMT and TDC response times.
Calibration corrections are determined using the LED source.
Each pulse of the LED leads to a calibration trigger so that all PMTs are read out.
The TDC value from each PMT is then converted to time using the
time step value of \SI{1.5625}{\ns} and adjusted
to account for the expected time of flight
of a photon produced in the center of the AD (where the LED is located)
to the given PMT.
After this adjustment, all PMTs should agree on the time of the LED pulse.
The time calibration correction is then computed
as the difference between an individual PMT's hit time
(adjusted for time of flight)
and the known pulse time of the LED.

\section{Channel quality}
\label{sec:channel_quality}

Occasionally one or more components of the PMT high voltage power supply
or data readout will malfunction or degrade,
leading to a substantial change in measured PMT gain.
Anomalous gain readings are flagged during weekly data quality checks,
and any problematic PMTs are disabled.
Usually at most one PMT at a time is affected in any AD.
The most common problem is a degraded high voltage supply,
and is fixed by replacing the high voltage module.
When any of the PMTs is disabled, the calorimetric response of the ADs
is slightly altered.
In particular, the number of photoelectrons collected
will decrease by $\nicefrac{n}{192}$ (on average),
relative to a configuration with \num{192} functioning PMTs,
with $n$ representing the number of disabled PMTs.
A correction is applied to the light yield to compensate for this average behavior.
\todo{Inves-tigate active channel correction}

%\section{Misc}

%The LED has a maximum wavelength of \SI{435}{\nm}
%and is used to calibrate the PMT gain and timing.
%The amplitude of the LED can be varied to measure the PMT response
%for different light levels.
%The LED pulse width is $\sim\SI{10}{\ns}$ to allow for
%precise measurements of differing response times among the PMTs.
%The specifics of the time and gain calibration are described below.

