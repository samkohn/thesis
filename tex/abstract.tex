\begin{abstract}
    A measurement of the smallest neutrino mixing angle, \thetaot{},
    by observations of reactor \nuebar{} disappearance
    at the Daya Bay Reactor Neutrino Experiment over 1958 days, is described.
    Eight identically-designed antineutrino detectors (ADs) monitored the \nuebar{} flux
    produced by six \SI{2.9}{\GW\of{th}} nuclear reactors,
    with four ADs located close to the reactors
    ($\sim$\SIrange[range-phrase = --]{350}{600}{\m})
    monitoring the unoscillated flux,
    and four ADs located far from the reactors
    ($\sim$\SIrange[range-phrase = --]{1500}{1950}{\m})
    observing the oscillated flux.
    Inverse beta decay events were identified based on the coincidence
    of the prompt positron annihilation and the delayed neutron capture on hydrogen
    within the organic liquid scintillator target of each AD.
    A selection based on both the time delay and distance
    between the prompt and delayed events
    allowed for a strong suppression of the largest background,
    uncorrelated (accidental) coincidences of decays of radiocontaminants in the ADs.
    Differences in detection efficiency between ADs
    were constrained to within \SI{0.66}{\percent},
    dominated by the AD-uncorrelated uncertainty
    of the coincidence distance and time criteria.
    Comparison of the near AD and far AD observations,
    with appropriate adjustments for
    detector livetimes, AD-reactor baselines, and backgrounds,
    revealed a disappearance signal
    with minimal dependence on reactor modeling and detector response.
    A $\chi^2$ expression
    with nuisance parameters was constructed to model the impact of \thetaot{}
    and systematic uncertainties on the predicted far AD observations
    given the near AD observations as input.
    A fit was performed using the exact three-flavor probability
    for \nuebar{} disappearance
    assuming the normal mass ordering,
    and found $\sin^22\thetaot = 0.0731^{+0.0087}_{-0.0089}$.
\end{abstract}
