\chapter{Neutrino Oscillations}
\label{ch:intro}

The study of neutrinos and their oscillations has been the source
of countless surprises in the development of the
Standard Model of particle physics and beyond.
Neutrinos are currently the only particles exhibiting properties
not explained by the Standard Model
which can be studied---not just searched for---on Earth in a laboratory setting.
This chapter will introduce the history of neutrinos (\cref{sec:history});
describe the modern theoretical understanding of neutrinos,
both in the Standard Model and beyond (\cref{sec:osc_intro});
and present the experimental techniques used to study neutrino oscillations,
focusing on those used by the Daya Bay experiment (\cref{sec:experiment_intro}).


\section{History of neutrinos}
\label{sec:history}

The first evidence of neutrinos emerged in the form of
the continuous spectrum of $\beta$ particles
emitted during nuclear decay.
The continuous spectrum was first reported by Chadwick in 1914 \cite{chadwick_beta}.
Additional measurements confirming Chadwick's result accumulated over the next decade.
At the time, $\beta$ decay was thought to be a 2-body decay:
\begin{equation}\label{eq:old_beta}
    N(A, Z) \to N(A, Z+1) + \beta^-
\end{equation}
Conservation of energy uniquely determines the energy
of the outgoing $\beta$ particles for a given decay,
thus producing a monoenergetic spectrum,
as had been observed for $\alpha$ and $\gamma$ decays.
In 1931, rather than sacrifice the principle of conservation of energy,
Pauli proposed the introduction of a light neutral particle
produced during $\beta$ decay that escaped detection
due to a low interaction cross section \cite{pauli_letter}.
The particle, eventually named the neutrino ($\nu$) by Fermi,
provided the additional degree of freedom needed
to maintain conservation of energy and allow for a continuous $\beta$ spectrum
via the 3-body decay
\begin{equation}\label{eq:beta_mid}
    N(A, Z) \to N(A, Z+1) + \beta^- + \nu
\end{equation}
Subsequent experiments, detailed below,
would reveal the existence of an antineutrino $\bar{\nu}$,
followed by a distinction between neutrino flavors $\nu_e, \nu_\mu,$
and, much later, $\nu_\tau$.
Additionally, the discoveries of the proton and neutron
led to the conclusion that $\beta$ decay was the result of
the decay of a single neutron rather than of the nucleus as a whole.
The modern understanding of $\beta$ decay is therefore
\begin{equation}\label{eq:beta_modern}
    n \to p + e^- + \nuebar
\end{equation}
This process and its permutations
provide the vast majority of opportunities for
experimental investigation of the neutrino to this day.

\subsection{Discovery of the neutrino}
\label{subsec:discovery}

Reines and Cowan were the first to observe the neutrino experimentally in 1956
\cite{reines_cowan}.
They used a liquid scintillator detector to monitor a target of
water with dissolved $\text{CdCl}_2$
placed near the Savannah River Plant nuclear reactor.
The neutrinos were expected to undergo the inverse beta decay (IBD) reaction,
\begin{equation}\label{eq:ibd}
    \nuebar + p \to n + e^+,
\end{equation}
with the proton supplied by a hydrogen nucleus (\isotope[1]{H}) in the water.
The positron would annihilate in the water almost immediately,
producing two $\gamma$ rays which were detected in the liquid scintillator
as a prompt signal.
The neutron would scatter within the target for a few ($\sim5$) \si{\us}
and eventually capture on a Cd nucleus,
triggering the emission of several $\gamma$ rays totalling \SI{9}{\MeV},
detected as a delayed event by the liquid scintillator.
IBD events were identified by the coincidence of a prompt and delayed event
at a rate of \SI[per-mode=reciprocal]{2.88\pm0.22}{\per\hour},
only present when the reactor was powered on.
This experiment was the progenitor of the Daya Bay experiment,
which also observes IBD events using a liquid scintillator detector
and the principle of the prompt-delayed coincidence.

\subsection{Difference between \texorpdfstring{$\nu$ and $\bar{\nu}$}{nu and nu-bar}}
\label{subsec:nu_vs_nubar}

Even prior to Reines and Cowan's observation of the neutrino,
Davis was able to provide convincing evidence that
the antineutrino was different from the neutrino \cite{davis_diff_nuebar}.
In 1954, Davis attempted to observe the reaction
\begin{equation}\label{eq:davis_nubar}
    \isotope[37]{Cl} + \bar{\nu} \to \isotope[37]{Ar} + \beta^-
\end{equation}
which is forbidden by lepton number conservation.
If $\nu=\bar{\nu}$, then lepton number conservation is violated
and the reaction could occur.
A tank containing \SI{3900}{\liter} of $\text{CCl}_4$
was placed in close proximity to the Brookhaven nuclear reactor.
Any \isotope[37]{Ar} produced was extracted using helium gas,
and the resulting gas was monitored using a Geiger counter
to observe the decay of \isotope[37]{Ar} (half-life 35 days).
No additional decays of \isotope[37]{Ar} were observed over the background rate.
Thus it was concluded that $\nu\neq\bar{\nu}$.

Davis noted in the conclusion of his report that the same experimental setup
could be used to detect neutrinos produced by the sun.
This experiment will be discussed later as the earliest evidence
for neutrino oscillations.

\subsection{Helicity of the neutrino}
\label{subsec:helicity}

The helicity $\frac{\sigma\cdot p}{\vert p\vert}$ of the neutrino was measured by
Goldhaber, Grodzins and Sunyar in 1957 \cite{helicity_measurement,helicity_review}.
The reaction chain
\begin{align}\label{eq:lampshade}
    \begin{split}
        e^- + \isotope[152m]{Eu}(0^-) \to \nu_e
        + & \isotope[152]{Sm}^*(1^-) \\
          & \isotope[152]{Sm}^*(1^-) \to \isotope[152]{Sm}(0^+) + \gamma
    \end{split}
\end{align}
allows for a connection between the helicity (polarization)
of the emitted $\gamma$
with the helicity of the $\nu_e$, which is not directly detected,
as follows.
The total inital momentum is approximately 0.
Therefore the decay products $\nu_e$ and $\isotope[152]{Sm}^*(1^-)$
must have equal and opposite momenta.
Similarly, the initial angular momentum (including spin) is $\pm\nicefrac{1}{2}$,
and the daughter nucleus is known to have spin 1,
therefore the spins of the decay products must point in opposite directions.
Since both decay products have spins and momenta in opposite directions,
they must have the same helicity.
When the $\isotope[152]{Sm}^*(1^-)$ decays,
occasionally the $\gamma$ will be emitted in the same direction
as the momentum of the parent nucleus (recoiling against the $\nu_e$).
In this case, since the final state of the \isotope[152]{Sm} has spin 0,
the $\gamma$ must have the same spin and direction of momentum
as the parent $\isotope[152]{Sm}^*$ and therefore the same helicity as the $\nu_e$.
If the $\gamma$ is emitted in a different direction,
it has a nonzero probability of having the opposite helicity.
A method was required to select only those $\gamma$ rays which
were emitted in the same direction as the momentum of the parent nucleus
to ensure they had the same helicity as the emitted $\nu_e$.

\begin{figure}
    \includegraphics[width=0.5\textwidth]{ch_introduction/helicity_setup}
    \caption{
        Schematic layout of the experiment which measured
        the helicity of the neutrino \cite{helicity_measurement}.
    }
    \label{fig:lampshade}
\end{figure}

A solution was found in the form of resonant fluorescence,
the absorption and re-emission of a $\gamma$ ray
by a nucleus with an energy level equal to that of the $\gamma$ ray.
An obvious choice for this setup was the exact same isotope of $\isotope[152]{Sm}$,
so target of $\text{Sm}_2\text{O}_3$
containing \SI{26.8}{\percent} \isotope[152]{Sm}
was placed in a ring below the sample of \isotope[152]{Eu}
(\cref{fig:lampshade}).
The $\gamma$ rays which resonantly scattered on the $\text{Sm}_2\text{O}_3$
were detected using a NaI scintillation counter.
This arrangement selected only the desired $\gamma$ rays
(those emitted in the same direction as the parent nucleus)
due to the following properties of resonant fluorescence:
The emitted $\gamma$ rays are in general
slightly (\SI{\sim3}{\eV}) below the energy
needed to re-excite the same mode in a target nucleus
due to the recoil during absorption.
The $\gamma$ rays emitted in exactly the same direction
as the parent $\isotope[152]{Sm}^*$
experince a recoil on emission which takes the form of
a Doppler shift to higher energy
(due to the parent's own recoil against the $\nu_e$).
Even the additional energy from the maximal Doppler shift
was not sufficient to cause the resonant fluorescence;
only when coupled with the random thermal motion
of both the parent and target nuclei
was there a non-negligible probability of interaction.
Thus the $\gamma$ rays reaching the NaI counter
must have been resonantly scattered off the target \isotope[152]{Sm};
therefore they must have been emitted in exactly the same direction
as the parent $\isotope[152]{Sm}^*(1-)$,
carrying the same helicity as the $\nu_e$
that caused the initial recoil.

The helicity of the emitted $\gamma$ rays was measured
using an electromagnet surrounding the \isotope[152]{Eu} source.
When the electrons in the magnet were polarized in a particular direction,
only $\gamma$ rays with the opposite polarization (spin)
would Compton scatter off them.
By comparing the number of resonant scatters
using an upward-pointing magnetic field
with the number from a downward-pointing field,
the polarization, and therefore the helicity, could be measured.
With the field pointing up, the polarized electron spins point down,
and therefore a downward-traveling $\gamma$ ray
would only scatter if it had a upward-pointing spin, i.e. helicity $-1$.
Conversely, downward-traveling $\gamma$ rays
with helicity $-1$ are preferentially transmitted
when the field is pointing down.
Only the transmitted $\gamma$ rays have the opportunity to
resonantly scatter off the $\text{Sm}_2\text{O}_3$
and be detected by the NaI counter.
A measured excess of
$\frac{2(N_{\downarrow}-N_{\uparrow})}{N_{\downarrow}+N_{\uparrow}} =
\num{0.017\pm0.003}$
was observed,
indicating that the $\gamma$ rays, and therefore the neutrino,
had helicity $-1$.


\subsection{Discovery of neutrino flavors}
\label{subsec:nu_flavors}

In 1962, Schwartz, Lederman and others used the
Alternating Gradient Synchrotron (AGS) at Brookhaven
to demonstrate that $\nu_\mu$ behaved differently from $\nu_e$
\cite{numu_vs_nue}.
In the world's first accelerator neutrino experiment,
they produced a beam of $\nu_\mu$ by directing
the \SI{15}{\GeV} AGS proton beam
onto a fixed beryllium target and relying on the following reaction chain:
\begin{align}\label{eq:accel_reaction_chain}
    \begin{split}
        p + \text{Be} \to &\pi^{\pm} + X \\
        &\pi^{\pm} \to \mu^{\pm} + \nu(\bar{\nu}) \\
    \end{split}
\end{align}
A series of spark chambers of total target mass \SI{10}{\tonne}
was exposed to the neutrino beam,
with the non-neutrino beam products attenuated
by a \SI{13.5}{\m} thick steel wall.
In a universe with only one type of neutrino (and one type of antineutrino),
then the neutrino beam would have produced equal quantities
of electrons and muons.
After an exposure of \num{3.48e17}~protons ($\sim300$~hours),
a total of 113 events were observed,
of which 56 were identified as muon-like single-track or vertex events
(5 of which were statistically attributed to cosmic ray contamination),
8 were electron-like electromagnetic showers, and 49 were background.
The electron-like events were attributed to
kaon contamination in the beam,
and, in any event, were not consistent with the
expected energy distribution of events due to the
primary muon-associated neutrino beam.
With dozens of muon events and not a single electron event
attributable to the muon-associated neutrinos,
it was concluded that $\nu_\mu \neq \nu_e$.


\subsection{Discovery of the neutral current}
\label{subsec:neutral_current}

In 1973, the Gargamelle neutrino experiment at CERN
published an observation of neutrino interactions
that did not involve a charged lepton,
thus proving the existence of the hadronic neutral current (NC) weak interaction
\cite{gargamelle_short,gargamelle}.
The Gargamelle bubble chamber contained \SI{\sim10}{\tonne} of
$\text{CF}_3\text{Br}$ within a \SI{2}{\tesla} magnetic field.
It was exposed to a neutrino beam which could switch
between $\nu_\mu$ and $\bar{\nu}_\mu$ modes.
A search was undertaken for events of the form
\begin{align}\label{eq:neutral_current}
    \begin{split}
        \nu_\mu(\bar{\nu}_\mu) + \text{N} &\to \nu_\mu(\bar{\nu}_\mu)
        + \text{hadrons, and} \\
        \nu_\mu(\bar{\nu}_\mu) + \text{N} &\to \mu^-(\mu^+) + \text{hadrons},
    \end{split}
\end{align}
with the former representing NC interactions
and the latter representing charged current (CC).
102 $\nu$- and 64 $\bar{\nu}$-induced interactions were observed
with the NC signature of only hadrons as interaction products,
and no associated $\mu$.

Gargamelle also observed a small number of leptonic NC interactions
of the form $\nu + e \to \nu + e$.
The first such event observed was found by a graduate student
in 1974.
Example bubble chamber photographs of hadronic
and leptonic NC interactions are shown in \cref{fig:gargamelle}.

\begin{figure}
    \centering
    \begin{subfigure}{0.49\textwidth}
        \includegraphics[width=\textwidth]{ch_introduction/gargamelle_hadronic}
    \end{subfigure}
    \begin{subfigure}{0.49\textwidth}
        \includegraphics[width=\textwidth]{ch_introduction/gargamelle_leptonic}
    \end{subfigure}
    \caption{
        Photographs of a hadronic NC interaction (left, \cite{gargamelle_leptonic})
        and the first observed leptonic NC interaction (right, \cite{gargamelle})
        taken by the Gargamelle neutrino experiment.
    }
    \label{fig:gargamelle}
\end{figure}

\subsection{The solar neutrino problem}
\label{subsec:homestake}

When Davis concluded the experiment which determined that $\nu\neq\bar{\nu}$,
he noted that the same experimental principle could be used
to search for neutrinos produced by the sun \cite{davis_diff_nuebar}.
This would provide experimental validation of the Standard Solar Model (SSM),
which describes the nuclear fusion processes
that are the source of the sun's energy.
The SSM predictions for neutrino fluxes (\cref{fig:solarflux})
were computed by Bahcall,
who worked closely with Davis \cite{bahcall2004}.
A tank of \SI{610}{\tonne} of $\text{CCL}_4$
located in the Homestake gold mine in South Dakota
was used as a target for the reaction
\begin{equation}\label{eq:davis_nu}
    \isotope[37]{Cl} + \nu \to \isotope[37]{Ar} + \beta^-.
\end{equation}
This reaction has a threshold of \SI{0.814}{\MeV} \cite{solar_review}
and is accessible primarily to neutrinos
produced via \isotope[8]{B} decay in the sun.
A steady supply of helium gas was fed through the tank
to collect the \isotope[37]{Ar},
which was then extracted from the helium,
separated from the contaminants xenon and krypton which were
also dissolved in the $\text{CCL}_4$,
and monitored with a proportional counter,
with each decay corresponding to a single neutrino interaction \cite{homestake1968}.
The experiment began in 1965, with results released periodically
until it shut down in 1992.

\begin{figure}
    \includegraphics[width=\textwidth]{ch_introduction/solar_spectrum}
    \caption{
        Predicted solar neutrino fluxes according to the SSM \cite{bahcall2004}.
    }
    \label{fig:solarflux}
\end{figure}

Even in the early published results,
the experiment observed a substantial deficit of neutrino events
compared to the prediction from the SSM.
Results were expressed in solar neutrino units (SNU),
with $\SI{1}{\SNU} = 1$ interaction per second per target nucleus.
In 1968, no significant number of events above background was observed;
the limit of no more than \SI{3}{\SNU} was compared to a prediction of
\SI{20\pm12}{\SNU} \cite{homestake1968}.
After more than a decade of observation,
the prediction and observation had evolved to \SI{5.8\pm2.2}{\SNU}
and \SI{2.1\pm0.3}{\SNU}, respectively,
and the discrepancy had become known as the ``solar neutrino problem.''
The two remaining routes for resolving the discrepancy were
(1) errors in the SSM, and (2) that neutrinos ``oscillate or decay''
between their production in the sun and their detection on Earth \cite{davis1985}.

Davis and others supported the construction of
new experiments based on the same radiochemical principle,
but using the reaction
\begin{equation}\label{eq:gallium}
    \isotope[71]{Ga} + \nu \to \isotope[71]{Ge} + \beta^-,
\end{equation}
which has a substantially lower threshold
of \SI{0.233}{\MeV} (cf. \SI{0.814}{\MeV} for \isotope[37]{Cl})
and thus is sensitive not just to \isotope[8]{B} neutrinos
but also those from \isotope[7]{Be} and the $pp$ and $pep$ fusion cycles.
The SAGE and GALLEX experiments observed deficits of approximately \SI{50}{\percent}
compared to the SSM prediction \cite{sage,gallex}.
The discrepancy between the \SI{\sim33}{\percent} deficit
in the \isotope[37]{Cl} experiment
and the \SI{50}{\percent} deficit in the \isotope[71]{Ga} experiments
added additional layers to the solar neutrino problem.
Separately, the Kamiokande-II experiment,
employing an entirely different detection concept
with a higher energy threshold,
also observed \SI{\sim50}{\percent} of the SSM-predicted solar neutrino flux.

In 1999, the Super-Kamiokande experiment's water cherenkov detector
observed a deficit of upward-going atmospheric $\nu_\mu$
relative to the rate of downward-going $\nu_\mu$.
The upward-going neutrinos were produced on the opposite side of the Earth
and so traveled farther than the downward-going neutrinos.
Since negligibly few neutrinos scattered as they propagated through the Earth,
this result proved that atmospheric $\nu_\mu$ oscillate to other flavors
as they travel,
suggesting that the solar neutrino problem may be resolved for $\nu_e$
through a similar mechanism. \todo{cite SuperK}
Super-Kamiokande also made a measurement of the solar neutrino flux
which matched its predecessor Kamiokande's measurement.

The Sudbury Neutrino Observatory (SNO) definitively resolved
the solar neutrino problem in 2001 in favor of neutrino oscillations,
validating the SSM as an accurate model of the processes powering the sun.
\todo{cite SNO}
The experiment monitored \SI{1}{\tonne} of heavy water ($\text{D}_2\text{O}$)
surrounded by a spherical array of photomultiplier tubes (PMTs)
which detect the Cherenkov radiation produced by neutrino interaction products.
The use of heavy water, first proposed by Chen in 1984, \todo{cite Chen D2O}
allowed SNO to observe both charged-current (CC)
and neutral-current (NC) interactions,
and thus compare the flux of solar $\nu_e$ with the total $\nu$ flux
from the sun over all flavors.
The three interaction categories for SNO were
\begin{align}\label{eq:sno_interactions}
    \begin{split}
        \nu_e + \isotope[2]{H} & \to e^- + p + p \text{ (CC)} \\
        \nu_x + \isotope[2]{H} & \to \nu_x + n + p
        \to \nu_x + \isotope[2]{H} + \gamma \text{ (NC)} \\
        \nu_x + e^- & \to \nu_x + e^- \text{ (elastic scatter (ES))}
    \end{split}
\end{align}
In these reactions $\nu_x$ means any flavor of neutrino ($x = e,\mu,\tau$).
ES reactions could occur via CC (for $\nu_e$ only, and with a higher cross section)
or via NC (for all flavors including $\nu_e$).
SNO reproduced the existing deficits using the CC interaction category,
corroborating the observation of missing solar $\nu_e$.
The NC observation that included all neutrino flavors matched the SSM prediction,
as did the ES measurement.
All of the neutrinos predicted by the SSM did reach Earth,
but only some of them interacted as $\nu_e$: thus their flavors had changed.
A summary diagram of the solar neutrino problem and its resolution by SNO
is shown in \cref{fig:solar_neutrino_fixed}.

\begin{figure}
    \includegraphics[width=\textwidth]{ch_introduction/solar_resolution}
    \caption{
        Predictions and measurements for \isotope[37]{Cl}, \isotope[71]{Ga},
        $\text{H}_2\text{O}$, and $\text{D}_2\text{O}$ solar neutrino experiments.
        Taken from \cite{bahcall_images} based on \cite{bahcall2005_diagram}.
    }
    \label{fig:solar_neutrino_fixed}
\end{figure}



\section{Neutrino oscillations}
\label{sec:osc_intro}

The Standard Model of particle physics (SM)
describes neutrinos as the massless electroweak partners
to the left-handed charged leptons $e_L,\mu_L,\tau_L$.
The discovery of neutrino oscillations provided proof that,
like for quarks, neutrino flavor eigenstates are not
eigenstates of the Hamiltonian (``energy eigenstates''),
and further, that neutrinos have nonzero mass.
The theoretical formalism for neutrino oscillation is discussed,
followed by a description of the experiments which measured
each oscillation parameter.

\subsection{Oscillation theory}
\label{subsec:theory}

In the following, $c = \hbar = 1$ unless otherwise noted.
In a generic theory with $n$ leptonic flavors,
each charged lepton $l_\alpha^-$ is paired with
an associated neutrino $\nu_\alpha$.
The association is realized via the charged current (CC) interaction vertex,
one form of which is $W^- \to l_\alpha^- + \nu_\alpha$.
The neutrino state participating in this interaction
is known as the $\alpha$ flavor eigenstate, notated $\ket{\nu_\alpha}$.
In general, the flavor eigenstates will not be equal to the energy eigenstates
$\ket{\nu_i}$
(which are eigenstates of the Hamiltonian and have definite mass).
The relation between the mass and flavor eigenstates can be written as
\begin{align}\label{eq:general_eigenstates}
    \begin{split}
        \ket{\nu_\alpha} &= \sum_{i=1}^n U^*_{\alpha i} \ket{\nu_i} \\
        \ket{\nu_i} &= \sum_{\alpha=1}^n U_{\alpha i} \ket{\nu_\alpha}
    \end{split}
\end{align}
where $U$ is a unitary matrix known as the mixing matrix.
The components of this equation obey the following unitarity relations:
\begin{align}\label{eq:unitarity}
    \begin{split}
        \braket{\nu_\alpha | \nu_\beta} &= \delta_{\alpha\beta} \\
        \braket{\nu_i | \nu_j} &= \delta_{ij} \\
        U^\dagger U &= \mathbf{1}
    \end{split}
\end{align}

The mass eigenstates are eigenstates of the Hamiltonian,
$\hat{H}\ket{\nu_i(t)} = E_i\ket{\nu_i(t)}$,
thus Schroedinger's equation takes a simple form:
\begin{align}\label{eq:schroedinger}
    \begin{split}
        \hat{H}\ket{\nu_i(t)} = E_i\ket{\nu_i(t)} &= i\frac{d}{dt}\ket{\nu_i(t)} \\
        \ket{\nu_i(t)} &= e^{-iE_it}\ket{\nu_i(0)}
    \end{split}
\end{align}
Defining $\ket{\nu_i} = \ket{\nu_i(0)}$ and $\ket{\nu_\alpha} = \ket{\nu_\alpha(0)}$,
the evolution of a flavor eigenstate can be expressed
using \cref{eq:general_eigenstates}:
\begin{align}\label{eq:state_evolution}
    \begin{split}
        \ket{\nu_\alpha(t)}
        &= \sum_{i=1}^n U^*_{\alpha i} e^{-iE_it}\ket{\nu_i} \\
        &= \sum_{i=1}^nU^*_{\alpha i} e^{-iE_it}
        \left(\sum_{\rho=1}^n U_{\rho i} \ket{\nu_\rho}\right) \\
        &= \sum_i\sum_\rho U^*_{\alpha i} U_{\rho i} e^{-iE_it}\ket{\nu_\rho}
    \end{split}
\end{align}
The amplitude $A_{\alpha\to\beta}(t)$
for observing the state $\ket{\nu_\beta}$ after
a neutrino produced as $\ket{\nu_\alpha}$ evolves over time $t$
is computed as the overlap
\begin{align}\label{eq:osc_amplitude}
    \begin{split}
        A_{\alpha\to\beta}(t) = \braket{\nu_\beta | \nu_\alpha(t)}
        &= \bra{\nu_\beta} \sum_i\sum_\rho U^*_{\alpha i} U_{\rho i}
        e^{-iE_it}\ket{\nu_\rho} \\
        &= \sum_i U^*_{\alpha i} U_{\beta i} e^{-iE_it} \\
    \end{split}
\end{align}
The probability for observing $\ket{\nu_\alpha(t)}$ in state $\ket{\nu_\beta}$
is simply the amplitude squared:
\begin{align}\label{eq:oscprob_general}
    \begin{split}
        P_{\alpha\to\beta}(t) = \left|A_{\alpha\to\beta}(t)\right|^2
        &= \left|\sum_i U^*_{\alpha i} U_{\beta i} e^{-iE_it}\right|^2 \\
        &= \left(\sum_iU_{\alpha i} U^*_{\beta i} e^{iE_it}\right)
        \left(\sum_j U^*_{\alpha j} U_{\beta j} e^{-iE_jt}\right) \\
        &= \sum_i\sum_j U^*_{\alpha j} U^*_{\beta i} U_{\alpha i} U_{\beta j}
        e^{-i(E_j - E_i)t}
    \end{split}
\end{align}
All (currently) experimentally-accessible neutrinos are ultrarelativistic,
i.e. $E \approx p \gg m$.
Thus the energy can be approximated as
\begin{align}\label{eq:energy_approx}
    \begin{split}
        E = \sqrt{p^2 + m^2}
        &= p\sqrt{1 + \frac{m^2}{p^2}} \\
        &\approx p\left(1 + \frac{m^2}{2p^2}\right) = p + \frac{m^2}{2p} \\
        &\approx E + \frac{m^2}{2E}
    \end{split}
\end{align}
At this point, a reasonable concern is the approximation $E \approx E + \frac{m^2}{2E}$
being self-referential, nearly tautological considering the assumption $E \gg m$.
A second leap of logic is the assertion that
$\ket{\nu_\alpha}$ was produced in a state of definite energy,
and thus $E_i$ equals the same constant $E$ for each superposed mass state $\ket{\nu_i}$,
motivating the identity
\begin{equation}\label{eq:msq_approx}
    E_j - E_i \approx \frac{m_j^2 - m_i^2}{2E}.
\end{equation}
The established framework for computing oscillations
using these approximations, known as the plane-wave formalism,
is applicable over an extraordinarily wide
kinematic parameter space despite the questionable logic.
A more general wave-packet formalism avoids these concerns
at the expense of additional mathematical machinery,
and reaches the same conclusions for all current experiments.
For a detailed discussion of the wave-packet formalism see \todo{cite wave packets}.

Continuing on by applying \cref{eq:msq_approx} to \cref{eq:oscprob_general}
with the ultrarelativistic approximation $t \approx L$
and the definition $\Delta m^2_{ij} = m_i^2 - m_j^2$,
\begin{align}\label{eq:oscprob_general_2}
    \begin{split}
        P_{\alpha\to\beta}(L)
        &= \sum_{ij} U^*_{\alpha j} U^*_{\beta i} U_{\alpha i} U_{\beta j}
        e^{i\frac{\Delta m^2_{ij}}{2E}L}
    \end{split}
\end{align}
This expression can be simplified by grouping the terms where $i=j$
and those where $i\neq j$.
When $i=j$, $\Delta m^2_{ij} = 0$.
For $i\neq j$, the matrix terms and the phase term can each be written
as the sum of real and imaginary parts and multiplied together term-by-term.
The final (real) probability contains contributions from both
the real and imaginary parts but as desired is purely real:
\begin{align}\label{eq:oscprob_general_3}
    \begin{split}
        P_{\alpha\to\beta}(L) =
        & \sum_i \left|U_{\alpha i}\right|^2 \left|U_{\beta i}\right|^2 \\
        & + 2\sum_{i>j} \Re \left[
            U^*_{\alpha j} U^*_{\beta i} U_{\alpha i} U_{\beta j}
        \right]
        \cos\left(\Delta m^2_{ij}\frac{L}{2E}\right) \\
        & + 2\sum_{i>j} \Im \left[
            U^*_{\alpha j} U^*_{\beta i} U_{\alpha i} U_{\beta j}
        \right]
        \sin\left(\Delta m^2_{ij}\frac{L}{2E}\right) \\
    \end{split}
\end{align}
with $\Re$ and $\Im$ denoting the real and imaginary parts (respectively)
of a complex number.
To simplify further, the leading term can be rewritten
using the unitarity relation, squared:
\begin{align}\label{eq:sub_unitarity}
    \begin{split}
        \delta_{\alpha\beta}
        &= (U^\dagger U)^2_{\alpha\beta} \\
        &= \sum_{ij} U^*_{\alpha j} U^*_{\beta i} U_{\alpha i} U_{\beta j} \\
        &= \sum_i \left|U_{\alpha i}\right|^2 \left|U_{\beta i}\right|^2
        + 2\sum_{i>j} \Re \left[
            U^*_{\alpha j} U^*_{\beta i} U_{\alpha i} U_{\beta j}
        \right]
    \end{split}
\end{align}
Substituting \cref{eq:sub_unitarity} into \cref{eq:oscprob_general_3}
and using the trigonometric identity $2\sin^2(\nicefrac{\theta}{2}) = 1 - \cos\theta$,
the general formula for oscillation probability for $n$ neutrinos is obtained:
\begin{align}\label{eq:oscprob_general_final}
    \begin{split}
        P_{\alpha\to\beta}(L) =
        &\ \delta_{\alpha\beta} \\
        & - 4\sum_{i>j} \Re \left[
            U^*_{\alpha j} U^*_{\beta i} U_{\alpha i} U_{\beta j}
        \right]
        \sin^2\left(\Delta m^2_{ij}\frac{L}{4E}\right) \\
        & + 2\sum_{i>j} \Im \left[
            U^*_{\alpha j} U^*_{\beta i} U_{\alpha i} U_{\beta j}
        \right]
        \sin\left(\Delta m^2_{ij}\frac{L}{2E}\right) \\
    \end{split}
\end{align}
In the special case where $\alpha = \beta$, the probability
is known as the survival probability.
The product of matrix elements
$U^*_{\alpha j} U^*_{\beta i} U_{\alpha i} U_{\beta j}$
becomes manifestly real,
hence the imaginary term in \cref{eq:oscprob_general_final} is zero,
yielding the general formula for survival probability:
\begin{equation}\label{eq:survival_prob_general}
        P_{\text{sur}} = P_{\alpha\to\alpha}(L) =
        1 - 4\sum_{i>j}
        \left|U_{\alpha i}\right|^2
        \left|U_{\alpha j}\right|^2
        \sin^2\left(\Delta m^2_{ij}\frac{L}{4E}\right) \\
\end{equation}
The phase of the oscillation can be rewritten in physical units,
inserting the dimensionful values for $c$ and $\hbar$:
\begin{equation}\label{eq:osc_phase_shorthand}
    \Delta m^2_{ij}\frac{L}{4E} \to
    \Delta m^2_{ij} c^4 \frac{L}{4\hbar cE}
    \approx 1.267
    \frac{\Delta m^2_{ij}}{(\SI{1}{\eV}/c^2)^2}
    \frac{L}{\SI{1}{\m}}
    \frac{\SI{1}{\MeV}}{E}
\end{equation}

When considering oscillations of antineutrinos ($\bar{\nu}$)
from (anti-)flavor $\bar{\alpha}$ to $\bar{\beta}$,
the only difference is that the complex conjugate
of the unitary mixing matrix is used,
leading to an identical result for survival probability,
and a slightly-altered result for the general case,
\begin{align}\label{eq:oscprob_general_anti}
    \begin{split}
        P_{\bar{\alpha}\to\bar{\beta}}(L) =
        &\ \delta_{\alpha\beta} \\
        & - 4\sum_{i>j} \Re \left[
            U^*_{\alpha j} U^*_{\beta i} U_{\alpha i} U_{\beta j}
        \right]
        \sin^2\left(\Delta m^2_{ij}\frac{L}{4E}\right) \\
        & - 2\sum_{i>j} \Im \left[
            U^*_{\alpha j} U^*_{\beta i} U_{\alpha i} U_{\beta j}
        \right]
        \sin\left(\Delta m^2_{ij}\frac{L}{2E}\right), \\
    \end{split}
\end{align}
where the application of the complex conjugate does not affect
the real part of the product of matrix terms,
and changes only the sign of the imaginary part.

\subsection{Two-neutrino mixing}

Before addressing the three-flavor oscillation phenomenology,
it is helpful to consider the two-flavor case,
which is both easier to understand
and also a good approximation to the oscillation behavior
of atmospheric $\nu_\mu$.
The $2\times2$ mixing matrix is purely real
and has only a single physical degree of freedom,
and has the form
\begin{equation}\label{eq:2d_mixing}
    U_{2\times2} =
    \begin{pmatrix}
        \cos\theta & \sin\theta \\
        -\sin\theta & \cos\theta
    \end{pmatrix},
\end{equation}
where $\theta$ is known as the mixing angle.
Since this matrix is purely real,
there is no difference between the neutrino and anti-neutrino
oscillation equations, preserving CP symmetry.
The two flavors are denoted $\alpha$ and $\beta$,
and the two mass states 1 and 2 with mass difference $\Delta m^2$.
The probability of a neutrino produced in the $\ket{\nu_\alpha}$ state
being observed as $\nu_\beta$ is then
\begin{equation}\label{eq:2d_osc}
    P_{\alpha\to\beta} = 4\cos^2\theta\sin^2\theta
    \sin^2\left(\Delta m^2_{ij}\frac{L}{4E}\right)
    = \sin^22\theta \sin^2\left(\Delta m^2_{ij}\frac{L}{4E}\right),
\end{equation}
and the survival probability is
\begin{equation}\label{eq:2d_p_sur}
    P_{\alpha\to\alpha} = 1 -
    \sin^22\theta \sin^2\left(\Delta m^2_{ij}\frac{L}{4E}\right),
\end{equation}
satisfying $P_{\alpha\to\beta} + P_{\alpha\to\alpha} = 1$ as expected.
These equations provide an intuitive meaning to the paramter $\theta$:
if $\theta = 0$, no oscillation will occur,
and if $\theta = \nicefrac{\pi}{2}$,
at the oscillation maximum, the neutrino will have a \SI{100}{\percent}
probability for being observed as a $\nu_\beta$.
This case is known as maximal mixing.
In the three-flavor scenario that describes the current understanding
of neutrino oscillation,
atmospheric $\nu_\mu$ are observed to mix near-maximally with $\nu_\tau$,
with very little probability to mix with $\nu_e$,
corresponding to the relevant mixing angle being close to $\nicefrac{\pi}{2}$.

\subsection{Three-neutrino mixing}

Three-neutrino mixing has a much more diverse phenomenology
than the simpler two-neutrino case
due to the additional degrees of freedom for a $3\times3$ unitary matrix.
The neutrino mixing matrix is known as the PMNS matrix,
named for Pontecorvo, Maki, Nakagawa and Sakata.
There are many different parametrizations of such a matrix,
but the one most convenient in describing neutrino oscillations is
\begin{equation}\label{eq:pmns}
    U_{\text{PMNS}} =
    \begin{pmatrix}
        1 & 0 & 0 \\
        0 & c_{23} & s_{23} \\
        0 & -s_{23} & c_{23}
    \end{pmatrix}
    \begin{pmatrix}
        c_{13} & 0 & s_{13}e^{-i\delta} \\
        0 & 1 & 0 \\
        -s_{13}e^{i\delta} & 0 & c_{13}
    \end{pmatrix}
    \begin{pmatrix}
        c_{12} & s_{12} & 0 \\
        -s_{12} & c_{12} & 0 \\
        0 & 0 & 1
    \end{pmatrix}
\end{equation}
where $s_{ij} = \sin\theta_{ij}$ for the three real mixing angles
$\theta_{12},\theta_{23}$, and $\theta_{13}$.
The complex phase $\delta$ is also known as $\delta_{CP}$
since CP violation is only possible if the PMNS matrix is complex,
i.e. $\delta_{CP} \notin \{0, \pi\}$.
The parametrization in \cref{eq:pmns} emphasizes the different oscillation regimes:
The first term is dominant for atmospheric neutrino oscillations,
with energies in the few \si{\GeV} range
and baselines of tens to thousands of \si{\km}.
The second term is dominant in reactor experiments
with energies in the few \si{\MeV} range
and baselines of \SI{\sim1}{\km},
and also in recent accelerator experiments
with both baselines and energies $\sim1000$ times higher.
The third term is dominant for solar neutrinos,
which, due to the interaction with the highly dense matter in the solar interior,
have an entirely different oscillation phenomenology from the one described here.
See \todo{Cite solar MSW} for a detailed discussion.

If the neutrino were a Majorana particle,
an additional two parameters, the Majorana phases,
would govern the mixing between states.
They would appear as an additional factor in \cref{eq:pmns}
of the form $\text{diag}(e^{i\alpha_1}, e^{i\alpha_2}, 1)$.
These additional parameters do not influence neutrino mixing.

Mass state 1 is defined as the lighter of the two states
participating in solar oscillations,
mass state 2 is the other relevant solar state,
and mass state 3 is the state which does not participate in solar oscillations.
The question of whether mass state 3 is heavier or lighter than the other two
is still unresolved;
this question is known as the neutrino mass ordering (or hierarchy) problem.
The normal ordering (NO) is $m_1 < m_2 < m_3 \Leftrightarrow \Delta m^2_{31} > 0$,
and the inverted ordering (IO) is
$m_3 < m_1 < m_2 \Leftrightarrow \Delta m^2_{31} < 0$.
No matter the true mass ordering,
the mass splittings of the three neutrinos obey the sum rule
\begin{equation}\label{eq:sum_rule}
    \Delta m^2_{31} = \Delta m^2_{21} + \Delta m^2_{32}.
\end{equation}
The current measured values for all oscillation parameters are \cite{pdg}
\todo{Fill in ``this work''}
\begin{align}\label{eq:current_values}
    \begin{split}
        \sin^2\theta_{12} &= \num{0.307\pm0.013} \\
        \sin^2\theta_{23} &=
        \begin{cases}
            \num{0.545\pm0.021} & (\text{NO}) \\
            \num{0.547\pm0.021} & (\text{IO})
        \end{cases} \\
        \sin^2\theta_{13} &=
        \begin{cases}
            \num{2.18\pm0.07e-2} & (\text{global average}) \\
                              & (\text{this work})
        \end{cases} \\
        \Delta m^2_{21} &= \SI{7.53\pm0.18e-5}{\eV\squared} \\
        \Delta m^2_{32} &=
        \begin{cases}
            \SI{2.453\pm0.034e-3}{\eV\squared} & (\text{NO}) \\
            -2.546^{+0.034}_{-0.040}\times 10^{-3}\,\si{\eV\squared} & (\text{IO})
        \end{cases} \\
        \delta_{CP} &= (\num{1.36\pm0.17})\times \SI{\pi}{\radian}\ (\text{NO})
    \end{split}
\end{align}
In this listing, the convention of reporting $\sin^2\theta_{ij}$
rather than $\sin^22\theta_{ij}$ was used
so that the octant of $\theta_{23}$
(whether it is greater than or less than \SI{45}{\degree})
is readily apparent.
At current experimental resolutions,
$\Delta m^2_{32}$ is still just barely indistinguishable from $\Delta m^2_{31}$,
given the above-quoted uncertainty in $\Delta m^2_{32}$ of
\SI{3.4e-5}{\eV\squared} compared to
a best-fit value for $\Delta m^2_{21}$ of \SI{7.53e-5}{\eV\squared}.

\begin{figure}
    \missingfigure{
        Plots showing oscillation probabilities when starting with
        $e,\mu,\tau$ neutrinos
    }
    \caption{Transition and survival probabilities for each neutrino flavor.}
    \label{fig:oscprob}
\end{figure}

\subsection{Measurement of oscillation parameters}
\label{subsec:osc_param_exp}

The solar parameters $\theta_{12}$ and $\Delta m^2_{21}$
were first measured using data from the aforementioned
Homestake / \isotope[37]{Cl}, SAGE, GALLEX/GNO, and SNO experiments.
Because the flavor transition mechanism is different from
the oscillation framework described in \cref{subsec:theory},
a detailed description of the measurement will be left to \cite{neutrino_textbook}.
Modern data relies on measurements from the Super-Kamiokande experiment.
The KamLAND experiment performed
an independent measurement of the solar parameters
by observing reactor antineutrinos
over a baseline of \SI{\sim285}{\km}.
The agreement of the reactor antineutrino measurement from KamLAND
with the existing solar neutrino results
was a major success of the theory of neutrino oscillations.

Though the earliest signs of neutrino disappearance and oscillations
were discovered in the solar neutrino sector,
a parallel history exists in observations of atmospheric neutrinos.
Atmospheric neutrinos are secondary decay products of
collisions of cosmic rays (mostly protons) with nuclei in the atmosphere.
Among the most common primary decay products are $\pi^{\pm}$,
which undergo the decay $\pi^+ \to \mu^+ + \nu_\mu$ and its CP conjugate.
The daughter muons then decay via $\mu^+ \to \bar{\nu_\mu} + e^+ + \nu_e$
and its CP conjugate.
Thus over the energy range where all muons decay before they reach the ground,
the expected ratio of $\nu_\mu$ to $\nu_e$ fluxes should be about 2.
The precise expectation can be computed with a Monte Carlo simulation
that incorporates knowledge of the atmosphere and of the kinematics of
pion, kaon and muon decays.

In 1988, the Kamiokande experiment observed a ratio of muon-type to electron-type
atmospheric neutrino events of only \SI{\sim60}{\percent}
as large as the Monte Carlo prediction,\todo{Cite Kamio\-kande atmo}
thus uncovering the so-called atmospheric neutrino anomaly.
The IMB experiment observed fluxes broadly consistent with
the reports from Kamiokande.
The Super-Kamiokande experiment provided the resolution
to the atmospheric anomaly in 1998
by comparing the flux of upward-going neutrinos
to that of downward-going neutrinos.
The fluxes were predicted to be independent of zenith angle
by a convenient property of spherical geometry
assuming only that the cosmic ray flux was uniform around the Earth,
which is true for cosmic rays with energy above a few \si{\GeV}.
Thus neutrino fluxes over different baselines
could be compared directly to each other
in a model-independent manner, rather than
just to Monte Carlo predictions.
A nonzero up-down asymmetry of $-0.296\pm0.048(\text{stat.})\pm0.01(\text{syst.})$
was observed for muon-type events (including neutrinos and antineutrinos),
but the up-down asymmetry for electron-type events
of $-0.036\pm0.076\pm0.02$ was consistent with zero \cite{superk1998}.
Later investigations \cite{superk2004} showed the characteristic $\nicefrac{L}{E}$
dependence when comparing the observed muon-type events
to a Monte Carlo prediction,
with the distance traveled inferred from the zenith angle.
\todo{Super-K figures}
The events used in the analysis had a reconstructed $\nicefrac{L}{E}$
with a resolution of \SI{<70}{\percent},
thus the characteristic shape of the $\nicefrac{L}{E}$ curve
was mostly washed out.
However, the location of the oscillation maximum
can be seen in Figure Y (where it appears as a minimum of surviving $\nu_\mu$),
providing the first measurement of $\Delta m^2_{32} = \SI{2.4e-3}{\eV\squared}$
Equally importantly, the asymptotic behavior
at large $\nicefrac{L}{E}$ reveals
the average value for the oscillation probability.
Since it is near $\nicefrac{1}{2}$,
the observation favored a value of $\sin^22\theta_{12} = 1$,
corresponding to maximal mixing.

The value of $\theta_{13}$ was first measured
by the Daya Bay experiment in 2012.
The experiment will be described in detail in the rest of this thesis.

The final oscillation parameter, $\delta_{CP}$,
is the subject of considerable current experimental effort.
Attempts to measure $\delta_{CP}$ involve
accelerator experiments comparing the rates of
$\nu_\mu\to\nu_e$ and $\bar{\nu}_\mu\to\bar{\nu}_e$ oscillations.
The current generation of experiments consists of
T2K, which utilizes the Super-Kamiokande detector;
and NOvA, which uses a large segmented liquid scintillator detector
which is the largest freestanding plastic structure ever built.
These experiments provided the above-referenced measurement to $\delta_{CP}$.
The next generation of experiments is currently under construction
and includes DUNE in the United States
and Hyper-Kamiokande in Japan.

Two classes of experiments are sensitive to the neutrino mass ordering.
The first is the same accelerator experiments searching for $\delta_{CP}$;
the neutrino beams in these experiments travel substantial distances
through the Earth's crust,
which induces different distortions to the $\nu_e$ appearance spectrum
depending on the mass ordering.
The second class of experiments currently consists of
a single experiment under construction: JUNO.
Using similar technology to Daya Bay,
the JUNO experiment will perform a precise measurement
of the reactor \nuebar{} spectrum
at the $\Delta m^2_{21}$ oscillation maximum,
where the slight difference between
$\Delta m^2_{32}$ and $\Delta m^2_{31}$
will cause different effects depending
on which mass splitting is larger.
Current experiments' data agrees better with normal ordering (NO)
predictions than with IO at the level of $2\sigma$ to $3\sigma$.

\section{Reactor neutrino experiments and \texorpdfstring{\thetaot}{theta13}}
\label{sec:experiment_intro}

As illustrated by \cref{fig:oscprob},
the probability of a transition of $\nu_e \leftrightarrow \nu_\mu$
or $\nu_e \leftrightarrow \nu_\tau$
has two characteristic wavelengths and amplitudes:
the larger is governed by $\theta_{12}$ and $\Delta m^2_{21}$,
while the smaller is governed by
$\theta_{13}$, $\Delta m^2_{32}$, and $\Delta m^2_{31}$
with a typical $\nicefrac{L}{E}$ scale of \SI{\sim0.5}{\km\per\MeV}.
The two experimentally-feasible signatures for \thetaot{}-governed oscillations
are $\nu_\mu\to\nu_e$ (appearance) and $\nu_e\to\nu_{\mu/\tau}$ (disappearance).
Appearance is most conveniently observed using accelerator neutrinos,
which are produced as a beam of primarily $\nu_\mu$ or $\bar{\nu}_\mu$.
Disappearance is easiest to observe using reactor \nuebar.

There are numerous systematic issues associated with the appearance measurement
which have until the last decade or so been prohibitive.
For example, since the probability of $\nu_e$ appearance is so small,
just a few percent,
the proportion of the beam that is either $\nu_\mu$ or $\nu_\tau$ is large,
necessitating extremely accurate flavor identification of detected events.
A small false positive rate of mistaking $\nu_\mu$ events as $\nu_e$
would substantially bias the measurement.
Further, the composition of the neutrino beam must be know extremely well.
The standard method for neutrino beam production (\cref{subsec:nu_flavors})
also produces charged kaons, which decay to $e^\pm + (\nu_e/\nuebar)$
with a much larger branching ratio than the equivalent decay for pions.
Thus a measurement of $\nu_e$ appearance must be able to justify
that it is not just observing $\nu_e$ that were produced directly
by the accelerator.

On the other hand, the disappearance measurement using reactor \nuebar is
free from many of the issues faced by the accelerator experiments.
With characteristic neutrino energies of \SIrange{1}{8}{\MeV},
any neutrinos which oscillated into $\nu_{\mu/\tau}$
would be below threshold for producing the associated charged lepton,
obviating the need for flavor identification.
Further, nuclear reactors produce neutrinos via $\beta^-$ decay,
thus producing only \nuebar.
The ideal interaction to observe these \nuebar events
is inverse beta decay (IBD),
which was used by Reines and Cowan in the first detection of neutrino events,
not coincidentally also from reactor \nuebar{} (\cref{subsec:discovery}).
As in the Reines and Cowan experiment,
modern reactor \nuebar{} experiments use liquid scintillator
as a detection medium.
Unlike in that earlier experiment, though,
modern experiments use liquid scintillator as the neutrino target itself,
using organic scintillators with a large number of free protons
in the form of \isotope[1]{H}.

The IBD reaction, $\nuebar + p \to e^+ + n$,
creates two signals with a characteristic
time delay between them, in a pattern known as a ``double''
or ``delayed'' coincidence.
The positron annihilates almost immediately (\SI{<1}{\nano\second})
into two $\gamma$-rays, which produce scintillation light
corresponding to the kinetic energy of the positron plus the
annihilation energy of $2\times \SI{0.511}{\mev}$.
This positron annihilation is the prompt signal
and serves as an effective timestamp for the interaction.
Meanwhile, the neutron thermalizes within $\sim \SI{10}{\micro\second}$
and then scatters randomly until it is eventually captured
on a nucleus in the target.
The neutron capture results in emission of one or multiple $\gamma$-rays,
which interact in the liquid scintillator to produce the delayed light signal.
This prompt-delayed coincidence pattern is an extremely efficient
discriminator for IBD events against a wide variety of sources of background.
The full IBD interaction and detection process
is depicted in \cref{fig:ibd_cartoon}.
Additional details on the specific properties of the Daya Bay
liquid scintillator and detector are provided in \cref{ch:detector}.

\begin{figure}
    \missingfigure{IBD cartoon}
    \caption{Graphical depiction of the IBD interaction and delayed coincidence.}
    \label{fig:ibd_cartoon}
\end{figure}

The first attempt to probe the \thetaot{} parameter space
in this way was the Chooz experiment in France. \todo{cite Chooz}
This experiment compared the observed number of \nuebar{} events
with the predicted number based on
models of the nuclear reactor \nuebar{} flux
and the detector response.
Chooz was unable to observe a value for \thetaot{} inconsistent with 0,
due to both low statistics and a large systematic uncertainty
stemming from the reactor modeling.

The next generation of experiments improved on Chooz
by increasing the target mass to get better statistics,
and by employing an additional set of detectors
located close to the reactor cores
that would constrain the un-oscillated flux.
By designing the near detectors to be functionally identical,
both sources the reactor modeling uncertainty
and the detector response uncertainty
would be nearly eliminated.
Three such experiments were built:
RENO in South Korea,
Double Chooz in France,
and Daya Bay in China, the subject of this thesis.

\section{Previous results}

Since the start of data taking on 24 December 2011,
the Daya Bay experiment has produced numerous measurements of
\thetaot{}, (searches for) sterile neutrinos, reactor \nuebar{} flux,
and a variety of other physical phenomena.
The first Daya Bay measurement of \thetaot{} in April 2012
was the first nonzero measurement of that quantity.
This measurement used \SI{55}{\day} of \nuebar{} data
and, as mentioned previously, only six ADs were operational at the time.
Since then, Daya Bay has published updated results using IBDs detected by
either neutron capture on Gadolinium (nGd)
(\cite{ngd2012,ngd2013,ngd2014,ngd2015,ngd2016,ngd2018}) or neutron capture on Hydrogen (nH),
as shown in \cref{fig:theta13_vs_t}.

\begin{figure}
    \centering
    \includegraphics[height=0.4\textheight]{ch_detector/theta13_vs_time}
    \caption{
        Published values of $\sin^{2}2\thetaot$ over time
        for both nGd and nH analyses.
        Some nGd results were reported with separate statistical
        and systematic errors;
        those have been combined linearly for this plot.
    }
    \label{fig:theta13_vs_t}
\end{figure}
