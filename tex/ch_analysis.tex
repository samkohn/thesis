\chapter{Measurement of \texorpdfstring{$\thetaot$}{theta13}}
\label{ch:analysis}

The 3-flavor model of neutrino oscillation described in \cref{ch:intro}
was tested for goodness-of-fit against the Daya Bay observations,
and best-fit values for the oscillation parameters \thetaot{} and \dmee{}
were extracted.
For a \nuebar{} with energy $E_\nu$,
the probability that it will be detected as a \nuebar{}
after traveling a distance $L$ is predicted in the 3-flavor model as:

\begin{align}\label{eq:p_sur}
    \begin{split}
        P_\text{sur} = 1 &- \cos^4\thetaot\sin^22\theta_{12}\sin^2\Delta_{21} \\
                         &- \sin^22\thetaot(\cos^2\theta_{12}\sin^2\Delta_{31}
                     + \sin^2\theta_{12}\sin^2\Delta_{32}) \\
        \simeq 1 &- \cos^4\thetaot\sin^22\theta_{12}\sin^2\Delta_{21} \\
                 &- \sin^22\thetaot\sin^2\Delta_{ee},
\end{split}
\end{align}
where
$\Delta_{ji} \simeq 1.267 \Delta m^2_{ji} (\si{\eV}^2) L(\si{\m})/E_\nu (\si{\MeV})$,
and
$\Delta m_{ee} \simeq \cos^2\theta_{12}\left|\Delta m^2_{31}\right| +
\sin^2\theta_{12}\left|\Delta m^2_{32}\right|$.
As described in detail in \cref{ch:intro},
the use of $\Delta m^2_{ee}$ to model the Daya Bay observations
is appropriate since the measurement is not sensitive
to the $O(\SI{1}{\percent})$ difference between $\Delta m^2_{31}$ and $\Delta m^2_{32}$.
Either form of \cref{eq:p_sur} can be used to compute the \nuebar{} survival probability
from the Daya Bay reactors to the near and far halls.
This analysis uses the second, simplified form.
Values for the physical mass-squared differences will be computed
based on the extracted value of $\Delta m^2_{ee}$.

To assess the validity of the 3-flavor model and extract best-fit values
of the oscillation parameters,
a comparison was made between the observed near-site and far-site \nuebar{} spectra.
The comparison relies on the number of observed events at the near halls (EH1 and EH2)
and $P_\text{sur}$ from the 3-flavor oscillation model
to predict the far-hall (EH3) observations
with minimal reliance on the intricacies of reactor modeling
and antineutrino production (\cref{sec:prediction}).
A \chisquare{} expression was created to quantify the agreement
between the model prediction and observations (\cref{sec:fitter}).

\section{Near-far projection}
\label{sec:prediction}

The Daya Bay experiment was configured to use
identically-designed antineutrino detectors (ADs) at near and far sites
so that the measurements of the near and far ADs could be directly compared
with minimal systematic uncertainty due to detection efficiency
and reactor modeling.
A predictive model was designed to implement the intuitive notion
that the observations at a near-hall AD can be used to predict
the observations at a far-hall AD \cite{p12e_fitter,p14a_fitter}.
This ``near-far projection'' model estimates the contribution of each reactor core
to a given near AD's IBD sample,
then extrapolates each core's contribution to the far hall
based on oscillation effects, AD-reactor distance,
and differences in livetime, target mass and efficiency.
An alternative class of models,
where the observations at both near and far ADs
are predicted based on a model of reactor \nuebar{} emission
in addition to oscillation effects,
has been used for both nH and nGd analyses \cite{nh2016, ngd2016}.
In this thesis, the near-far projection model is used for the first time
to extract \thetaot{} from observations of neutron capture on Hydrogen.


In the simplest configuration, with a single near observation,
a single far observation, and a single isotropic source of \nuebar,
the ratio of the number of observed far events $N_\text{f}$
to the number of observed near events $N_\text{n}$
depends on only a small set of quantities:
the detection efficiencies $\varepsilon_\text{n/f}$,
the number of target protons $N_\text{p,n/f}$,
the distances (baselines) $L_\text{n/f}$ between the detectors and the source,
and the oscillation survival probability $P_\text{sur}(E_\nu, L_\text{n/f})$.
If the efficiencies, baselines, and number of target protons are well-understood,
then the near-far ratio becomes sensitive
to small changes in the survival probability via the formula \cite{ngd2016}

\begin{equation}\label{eq:near_far}
    \frac{N_\text{f}}{N_\text{n}} = \left(\frac{N_\text{p,f}}{N_\text{p,n}}\right)
    \left(\frac{L_\text{n}}{L_\text{f}}\right)^2
    \left(\frac{\varepsilon_\text{f}}{\varepsilon_\text{n}}\right)
    \left[\frac{P_\text{sur}(E_\nu, L_\text{f})}{P_\text{sur}(E_\nu, L_\text{n}}\right].
\end{equation}

In practice, the presence at Daya Bay of multiple \nuebar{} sources
in the form of six reactor cores located hundreds of meters apart
necessitated a more complex variant of \cref{eq:near_far}
that accounted for both the differences in reactor power and spectrum over time,
and the two near halls, both of which observe \nuebar{}'s
in various stages of oscillation from all six reactor cores.
Given a set of observed IBD candidates at a near-hall Daya Bay AD,
the near-far projection model performs the following steps,
illustrated in \cref{fig:near_far_cartoon}:

\begin{figure}
    \missingfigure{Near-far cartoon}
    \caption{Illustration of the near-far projection model.}
    \label{fig:near_far_cartoon}
\end{figure}

\begin{enumerate}
    \item Subtract backgrounds from the near hall measurement
        and adjust for detection efficiencies
    \item Convert the reconstructed prompt energy spectrum
        into an estimated true \nuebar{} energy spectrum
    \item Predict the contribution of each reactor core
        the observed \nuebar{} spectrum,
        including minor oscillation effects
    \item Extrapolate each core's contribution to the far hall ADs,
        accounting for oscillation effects and the AD-reactor baseline
    \item Convert the \nuebar{} energy back into reconstructed energy
    \item Correct for the far AD's detection efficiency, backgrounds,
        target mass, and livetime
\end{enumerate}
The predicted spectrum at each far-hall AD can be compared
with the observed spectrum, as described in \cref{sec:fitter}.
For the rate-only analysis, a single bin of reconstructed energy is used.

\subsection{Near-hall backgrounds and efficiencies}
\label{subsec:near_bg_eff}

The observed number of IBD candidates must be adjusted
to account for the backgrounds and efficiencies described in \cref{ch:event_selection}.
The backgrounds this analysis accounts for are
accidentals, \li{}/\he{}, fast neutrons, and \amc{}.
The only efficiencies which vary significantly between ADs,
the muon-veto livetime efficiency $\varepsilon_\mu$
and the multiplicity efficiency $\varepsilon_m$,
can be measured precisely and so are directly accounted for.
Other efficiencies such as the delayed energy cut efficiency
are estimated for a generic or average AD and assigned a relative uncertainty.
Since the near-far projection model is a relative measurement,
only deviations from the estimated value
impact the final prediction.
Therefore the remaining efficiencies are accounted for
by terms which quantify that potential difference.
Given a near-hall observation of $N_{\text{cand},i}$ IBD candidates
and $N_{\text{bg},i}$ predicted backgrounds in reconstructed bin $i$,
the corrected number of IBDs is

\begin{equation}
    N_{\text{IBD},i} =
    \frac{N_{\text{cand},i} - N_{\text{bg},i}}{\varepsilon_\mu\varepsilon_m\prod_n(1+\nu_n)},
\end{equation}
where $\nu_n$ is the relative difference between efficiency $n$
and the estimated average value determined in \cref{ch:event_selection}.
In practice the $\nu_n$ are implemented as nuisance or pull parameters
so they can be adjusted during the fit procedure.

\subsection{Extracting the true \texorpdfstring{\nuebar{}}{antineutrino} spectrum}
\label{subsec:reco_to_true_energy}

The true \nuebar{} energy spectrum observed by the near halls
is required to compute oscillation probabilities.
The conversion from reconstructed to true energy
is impacted not only by the energy resolution and calibration
but also by a set of nonlinearities described in \cref{subsec:abs_energyscale}.
A detector response matrix was created
using the detailed Monte Carlo simulation (\cref{sec:thu_toymc})
by binning simulated events based on the true incoming \nuebar{} energy
and the reconstructed energy of the prompt event.
For each bin of reconstructed energy,
a probability distribution function (PDF) was constructed
which described the spectrum of incident \nuebar{}'s
attributable to the IBDs in that reconstructed energy bin.
The detector response matrix and the set of PDFs
for both the rate-only and spectral measurements
are shown in \cref{fig:drm}.
A separate true \nuebar{} spectrum is computed
for each bin of reconstructed energy for each near-hall AD as

\begin{equation}
    N_i(E_{\text{true}}) = N_{\text{IBD},i} \cdot f_{\text{DRM},i}(E_{\text{true}}),
\end{equation}
where $f_{\text{DRM},i}(E_{\text{true}})$ is the probability
that an event in reconstructed energy bin $i$
was caused by a \nuebar{} with true energy $E_{\text{true}}$.

This methodology was used instead of the more-obvious
inverting of the detector response matrix
because the process of inverting the matrix is numerically unstable.
Since variations in the detector response were highly constrained,
\todo{cite energy scale}
the resulting true \nuebar{} spectra were meaningful when compared between ADs.


\begin{figure}
    \missingfigure{Detector response matrix and normalized PDFs}
    \label{fig:drm}
\end{figure}










\section{Statistical methods}
\label{sec:fitter}

Standard frequentist techniques were used to determine best-fit parameters
and goodness-of-fit between the 3-flavor neutrino oscillation model
and the data observed by Daya Bay.
A \chisquare{} expression with nuisance parameters
was the primary tool for this analysis.
Variants of this technique have been used both in previous Daya Bay nH analyses
and in other Daya Bay results including nGd \thetaot{} analyses
and absolute reactor \nuebar{} flux and spectral measurements
\cite{nh2016,ngd2016,reactorflux2017,extractionreactorflux2019}.
The generic structure of such \chisquare{} expressions
can be derived from Eq.~(39.17) of \cite{pdg} as:

\begin{equation}
    \label{eq:chisquare_generic}
    \chisquare = \sum_i \left(
        \frac{F_{\text{obs},i} - F_{\text{pred},i}(\boldsymbol{\eta};\boldsymbol{\nu})}
            {\sigma_{\text{obs},i}}
        \right)^2
        +
        \sum_j \frac{\nu_j^2}{\tilde{\sigma}_{\nu,j}^2},
\end{equation}
where $F_{\text{obs},i}$ is the observed value for data point $i$
with uncertainty $\sigma_{\text{obs},i}$;
and $F_{\text{pred},i}(\boldsymbol{\eta};\boldsymbol{\nu})$
is the model prediction for data point $i$
which depends on $\boldsymbol{\eta}$, the model parameters of interest,
and $\boldsymbol{\nu}=(\nu_1, \ldots, \nu_m)$,
the model nuisance parameters.
In this formulation, the nuisance parameters are dimensionless,
and in the model they always accompany a paired quantity $A_j$,
which intuitively represents the physically-relevant quantity
(such as a detection efficiency or background rate)
that the nuisance parameter introduces uncertainty for:

\begin{equation}
    (1+\nu_j)A_j.
\end{equation}
Thus the $\nu_j$ can be interpreted as the deviation of $A_j$
from an expected or constrained value,
and the $\tilde{\sigma}_{\nu,j}$ are the (independently-determined)
\emph{relative} uncertainties of $A_j$
which set the scale for allowable values of $\nu_j$.
The $A_j$ remain fixed during the minimization procedure.

The values of $\boldsymbol{\eta}$ and $\boldsymbol{\nu}$
which minimize \cref{eq:chisquare_generic}
provide best-fit parameters of interest,
while ensuring that the nuisance parameters $\nu_j$
remain in some sense small.
Intuitively, the parameters $A_j$ are allowed to deviate slightly
from their estimated values,
but large deviations penalize the \chisquare{} and so are discouraged.

For this analysis, each data point $F_{\text{obs},i}$
is the number of observed events passing all event selection criteria
(\cref{ch:event_selection})
in a particular far-hall antineutrino detector $d$ (in EH3),
during a particular data-taking period $k$ (6-AD, 8-AD or 7-AD).
Since EH3-AD4 was not operative during the 6-AD period,
there are $3 + 4 + 4 = 11$ observed data points:\todo{energy bins?}

\begin{equation}
    \mathbf{F}_{\text{obs}} =
    (N^{\text{obs}}_{\text{6-AD, EH3-AD1}}, \ldots, N^{\text{obs}}_{\text{7-AD, EH3-AD4}})
\end{equation}
The model prediction
$\mathbf{F}_{\text{pred}}(\boldsymbol{\eta};\boldsymbol{\nu})$
of the event counts for the far-hall ADs,
described in depth in the following section,
depends on the observed event rates at the near halls (EH1 and EH2),
reactor power and fission fractions,
differing background rates for each AD,
conversion between reconstructed prompt energy and true \nuebar{} energy,
detection efficiencies,
and, of course, neutrino oscillation parameters.
The $\boldsymbol{\eta}$ vector contains the neutrino oscillation parameters.
The remaining dependencies are assigned to nuisance parameters
to account for uncertainties in the values used in the prediction.
In particular,

\begin{align}
    \begin{split}
        \boldsymbol{\eta} &= (\thetaot, \dmee) \\
        \boldsymbol{\nu} &= (
            \boldsymbol{\alpha},
            \boldsymbol{\epsilon},
            \boldsymbol{\eta_B},
            \boldsymbol{\eta_N}),
    \end{split}
\end{align}
where the nuisance parameters are collected into smaller vectors.
The vector $\boldsymbol{\alpha}$ represents reactor flux uncertainties,
$\boldsymbol{\epsilon}$ represents detection efficiency uncertainties,
$\boldsymbol{\eta_B}$ represents background rate uncertainties,
and $\boldsymbol{\eta_N}$ represents the statistical uncertainty
of the observed event rate at the near hall ADs.

The full \chisquare{} expression used in the rate-only\todo{rate + shape?} analysis is:

\begin{align}
    \begin{split}
        \chi^2 &= \sum_{\substack{j \in \\\text{far ADs}}}
            \frac{
                (N_{\text{obs},j}
                - N_{\text{pred},j}(\boldsymbol{\eta};\boldsymbol{\nu}))^2}
            {\sigma_{\text{obs},j}^2 } \\
            &+ \sum_{\substack{r \in \\\text{reactors}}}
                \frac{\alpha_r^2}{\tilde{\sigma}_R^2}
            + \sum_{\substack{d \in \\\text{all ADs}}}
            \left(
                \frac{\epsilon_d^2}{\tilde{\sigma}_D^2}
                + \frac{\eta_{B,d}^2}{\tilde{\sigma}_{B,d}^2}
            \right)
            + \sum_{\substack{d \in \\\text{near ADs}}}
            \frac{\eta_{N,d}^2}{\tilde{\sigma}^2_{\text{obs},d}}
    \end{split}
\end{align}

Fitting software was implemented using the SciPy package's
$\mathtt{optimize.least\_squares}$ function,
which relies on the Trust Region Reflective algorithm \todo{cite SciPy and TRR alg.}
to perform the least-squares fit.

%If the predictions based on different near-hall AD observations agree,
%it can be concluded that the effects due to reactor modeling
%and variation between detectors have been properly accounted for.



%For example, the number of accidental background events in a given near AD
%is estimated to be $N_{\text{acc}}$.
%In the model, this value is subtracted from the total
%observed events in that AD, $N_{\text{obs}}$.
%The expression used to account for the subtraction is,
%in simplified form,

%\begin{equation}
    %(1+\nu_{\text{obs}})N_{\text{obs}} - (1+\nu_{\text{acc}})N_{\text{acc}},
%\end{equation}
%thereby modeling both the statistical uncertainty of the near AD observation
%(as $\nu_{\text{obs}}$ is allowed to change)
%and the systematic uncertainty inherent in
%estimating the number of accidental background events
%(reflected in changes to $\nu_{\text{acc}}$).


