\chapter{Measurement of \texorpdfstring{$\thetaot$}{theta13}}
\label{ch:analysis}

The 3-flavor model of neutrino oscillation described in \cref{ch:intro}
was tested for goodness-of-fit against the Daya Bay observations,
and best-fit values for the oscillation parameters \thetaot{} and \dmee{}
were extracted.
A \chisquare{} expression was created to quantify the agreement
between the model prediction and observations (\cref{sec:fitter}).
The model prediction relies on the observed events at the near halls (EH1 and EH2)
to predict the far-hall (EH3) observations
with minimal reliance on the intricacies of reactor modelling
and neutrino production (\cref{sec:prediction}).

\section{Statistical methods}
\label{sec:fitter}

Standard frequentist techniques were used to determine best-fit parameters
and goodness-of-fit between the 3-flavor neutrino oscillation model
and the data observed by Daya Bay.
A \chisquare{} expression with nuisance parameters
was the primary tool for this analysis.
Variants of this technique have been used both in previous Daya Bay nH analyses
and in other Daya Bay results including nGd \thetaot{} analyses
and absolute reactor \nuebar{} flux and spectral measurements
\cite{nh2016,ngd2016,reactorflux2017,extractionreactorflux2019}.
The generic structure of such \chisquare{} expressions
can be derived from Eq.~(39.17) of \cite{pdg} as:

\begin{equation}
    \label{eq:chisquare_generic}
    \chisquare = \sum_i \left(
        \frac{F_{\text{obs},i} - F_{\text{pred},i}(\boldsymbol{\eta};\boldsymbol{\nu})}
            {\sigma_{\text{obs},i}}
        \right)^2
        +
        \sum_j \frac{\nu_j^2}{\tilde{\sigma}_{\nu,j}^2},
\end{equation}
where $F_{\text{obs},i}$ is the observed value for data point $i$
with uncertainty $\sigma_{\text{obs},i}$;
$F_{\text{pred},i}(\boldsymbol{\eta};\boldsymbol{\nu})$
is the model prediction for data point $i$
which depends on $\boldsymbol{\eta}$, the model parameters of interest,
and $\boldsymbol{\nu}=(\nu_1, \ldots, \nu_m)$,
the model nuisance parameters.
In this formulation, the nuisance parameters are dimensionless,
and in the model they always accompany a paired quantity $A_j$,
which intuitively represents the physically-relevant quantity
(such as a detection efficiency or background rate)
that the nuisance parameter introduces uncertainty for:

\begin{equation}
    (1+\nu_j)A_j.
\end{equation}
Thus the $\nu_j$ can be interpreted as the deviation of $A_j$
from an expected or constrained value,
and the $\tilde{\sigma}_{\nu,j}$ are the (independently-determined)
\emph{relative} uncertainties of $A_j$
which set the scale for allowable values of $\nu_j$.
The $A_j$ remain fixed during the minimization procedure.

The values of $\boldsymbol{\eta}$ and $\boldsymbol{\nu}$
which minimize \cref{eq:chisquare_generic}
provide best-fit parameters of interest,
while ensuring that the nuisance parameters $\nu_j$
remain in some sense small.
Intuitively, the parameters $A_j$ are allowed to deviate slightly
from their estimated values,
but large deviations penalize the \chisquare{} and so are discouraged.

For this analysis, each data point $F_{\text{obs},i}$
is the number of observed events passing all event selection criteria
(\cref{ch:event_selection})
in a particular far-hall antineutrino detector $d$ (in EH3),
during a particular data-taking period $k$ (6-AD, 8-AD or 7-AD).
Since EH3-AD4 was not operative during the 6-AD period,
there are $3 + 4 + 4 = 11$ observed data points:\todo{energy bins?}

\begin{equation}
    \mathbf{F}_{\text{obs}} =
    (N^{\text{obs}}_{\text{6-AD, EH3-AD1}}, \ldots, N^{\text{obs}}_{\text{7-AD, EH3-AD4}})
\end{equation}
The model prediction
$F_{\text{pred},i}(\boldsymbol{\eta};\boldsymbol{\nu}) = N_{\text{pred},j}(\boldsymbol{\eta};\boldsymbol{\nu})$,
described in depth in the following section,
depends on the observed event rates at the near halls (EH1 and EH2),
reactor power and fission fractions,
differing background rates for each AD,
conversion between reconstructed prompt energy and true \nuebar{} energy,
detection efficiencies,
and, of course, neutrino oscillation parameters.
The $\boldsymbol{\eta}$ vector contains the neutrino oscillation parameters.
The remaining dependencies are assigned to nuisance parameters
to account for uncertainties in the values used in the prediction.
In particular,

\begin{align}
    \begin{split}
        \boldsymbol{\eta} &= (\thetaot, \dmee) \\
        \boldsymbol{\nu} &= (
            \boldsymbol{\alpha},
            \boldsymbol{\epsilon},
            \boldsymbol{\eta_B},
            \boldsymbol{\eta_N}),
    \end{split}
\end{align}
where the nuisance parameters are collected into smaller vectors.
The vector $\boldsymbol{\alpha}$ represents reactor flux uncertainties,
$\boldsymbol{\epsilon}$ represents detection efficiency uncertainties,
$\boldsymbol{\eta_B}$ represents background rate uncertainties,
and $\boldsymbol{\eta_N}$ represents the statistical uncertainty
of the observed event rate at the near hall ADs.

The full \chisquare{} expression used in the rate-only\todo{rate + shape?} analysis is:

\begin{align}
    \begin{split}
        \chi^2 &= \sum_{\substack{j \in \\\text{far ADs}}}
            \frac{
                (N_{\text{obs},j}
                - N_{\text{pred},j}(\boldsymbol{\eta};\boldsymbol{\nu}))^2}
            {\sigma_{\text{obs},j}^2 } \\
            &+ \sum_{\substack{r \in \\\text{reactors}}}
                \frac{\alpha_r^2}{\tilde{\sigma}_R^2}
            + \sum_{\substack{d \in \\\text{all ADs}}}
            \left(
                \frac{\epsilon_d^2}{\tilde{\sigma}_D^2}
                + \frac{\eta_{B,d}^2}{\tilde{\sigma}_{B,d}^2}
            \right)
            + \sum_{\substack{d \in \\\text{near ADs}}}
            \frac{\eta_{N,d}^2}{\tilde{\sigma}^2_{\text{obs},d}}
    \end{split}
\end{align}

%For example, the number of accidental background events in a given near AD
%is estimated to be $N_{\text{acc}}$.
%In the model, this value is subtracted from the total
%observed events in that AD, $N_{\text{obs}}$.
%The expression used to account for the subtraction is,
%in simplified form,

%\begin{equation}
    %(1+\nu_{\text{obs}})N_{\text{obs}} - (1+\nu_{\text{acc}})N_{\text{acc}},
%\end{equation}
%thereby modeling both the statistical uncertainty of the near AD observation
%(as $\nu_{\text{obs}}$ is allowed to change)
%and the systematic uncertainty inherent in
%estimating the number of accidental background events
%(reflected in changes to $\nu_{\text{acc}}$).



\section{Near-far projection}
\label{sec:prediction}
