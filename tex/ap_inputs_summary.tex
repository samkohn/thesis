\chapter{Summary of analysis inputs}
\label{ap:inputs_summary}

Each quantity and measurement used as an input to the fitter
was constrained through individual studies.
A complete list of inputs and description of the constraints and validations
is provided below.

\section{Reactor}
\label{sec:summary_reactor}

Reactor operation data was provided by the power plant operator
and compiled into meaningful inputs by Daya Bay collaborators.
\begin{itemize}
    \item The reactor power levels were compiled directly from
        the supplied data.
    \item The reactor \nuebar{} emitted spectra and rates
        were computed using the Huber \cite{reactor_huber}
        and Mueller \cite{reactor_mueller} models,
        fission fractions provided by the power plant,
        and energy per fission from \cite{thermal_fission}.
\end{itemize}
These quantities were combined to produce the total reactor flux
as a function of energy used in \cref{subsec:flux_fraction,subsec:extrapolation}.
The total reactor flux summed over energy is listed in \cref{tab:total_emitted}.

\section{Antineutrino Detectors}
\label{sec:summary_ads}

The Daya Bay antineutrino detectors (ADs) were designed to be
nearly identical.
The major systematic uncertainties of the analysis derive from
the degree to which the ADs deviate from being exactly identical.

\begin{itemize}
    \item The detector response matrix was estimated using the
        individual event toy Monte Carlo described in \cref{sec:thu_toymc}.
        The simulation included effects from
        the absolute energy nonlinearity (\cref{subsec:abs_energyscale})
        and the energy resolution (\cref{subsec:resolution}).
        During the validation of the fitter on simulated nGd data sets
        in \cref{subsec:fitter_validation},
        the results using the nominal detector response
        and an independently-generated detector response from \cite{lbnl_toymc}
        were compared, and a negligible impact was observed.
    \item The expected true IBD spectrum was computed from
        the predicted reactor spectrum,
        the IBD cross section \cite{ibd_xsec},
        and oscillation effects (adjusted during fitting).
        Since the cross section is identical at all ADs,
        there was negligible uncertainty due to the cross section.
    \item The relative energy scale uncertainty was measured
        by fitting the nH capture delayed energy peaks
        using the calorimeter function (\cref{subsec:delayed}).
        The full range of fit values was \SI{0.5}{\percent},
        which was taken to be the relative energy scale uncertainty.
        %The impact of the relative energy scale uncertainty
        %was estimated using a toy Monte Carlo event sample \cref{sec:thu_toymc}.
        %The fractional change to the number of IBDs
        %within each bin of reconstructed energy was computed
        %assuming an \SI{0.5}{\percent} shift in relative energy scale.
        %During the fit, the pull parameter for a given AD
        %was used to adjust the event counts in each bin
        %based on the pre-computed fractional changes.
    \item The prompt energy cut efficiency uncertainty
        was estimated using a toy Monte Carlo event sample (\cref{sec:thu_toymc}).
        By shifting the reconstructed energies according to the
        \SI{0.5}{\percent} relative energy scale uncertainty,
        the change in efficiency could be measured to be \SI{0.1}{\percent}.
    \item The corrections to the prompt efficiency
        due to the oscillation-induced shape distortion
        were estimated using a toy Monte Carlo event sample (\cref{sec:thu_toymc}).
        The prompt efficiency was computed after re-weighting the sample
        based on a variety of combinations of oscillation parameters
        (\cref{subsec:prompt_energy}).
        During the fit, the exact correction factors were determined
        using a linear interpolation between pre-computed points.
    \item The delayed energy cut efficiency uncertainty
        was measured from data by comparing the relative increase in efficiency
        across ADs between the nominal cut and an extended cut
        (\cref{subsec:delayed}).
        The increase in efficiency for each AD was consistent to within
        \SI{\pm 0.2}{\percent} (half-range).
    \item The combined distance-time (DT) cut efficiency uncertainty
        was estimated using two different samples of accidentals-subtracted events:
        the nGd sample and an nH sample with a restricted prompt energy cut
        of $E_p > \SI{3.5}{\MeV}$ (\cref{sec:DT_cut}).
        Both of these samples were minimally impacted by the accidental background,
        so the accidentals-subtracted distributions
        had negligible systematic uncertainty.
        The absolute efficiencies for these samples were measured directly from data,
        and the half-range across ADs for each sample
        was used as the relative uncertainty.
        The AD-uncorrelated uncertainty for the final data sample
        was taken to be the larger of the relative uncertainties
        of the two samples.
    \item The number of target protons was computed using measured values
        of the masses of the GdLS, LS and acrylic detector volumes
        and the proton densities of each substance (\cref{subsec:target_mass}).
        The AD-uncorrelated uncertainty was computed
        using the uncertainties for the mass and proton density.
        A single relative uncertainty of \SI{0.37}{\percent} was used for all ADs.
    \item The muon veto efficiency ($\varepsilon_\mu$) was computed from data
        as the fraction of DAQ livetime that did not lie within a muon veto window
        (\cref{sec:muonveto}).
        The AD-uncorrelated uncertainty was negligible.
    \item The multiplicity veto efficiency ($\varepsilon_m$) was computed
        using the singles rate for each AD (\cref{sec:coincidence}).
        The AD-uncorrelated uncertainty was negligible.
\end{itemize}

\section{Backgrounds}
\label{sec:summary_bg}

Irreducible backgrounds comprised over \SI{50}{\percent} of the \nuebar{} candidates
at the far ADs.
The dominant background was the accidental (uncorrelated) background,
with smaller contributions from the correlated backgrounds.
The characterization of correlated backgrounds was performed by Daya Bay collaborators.

\begin{itemize}
    \item The accidental background rate was characterized using
        the singles (uncorrelated event) rate
        and a synthetic sample of accidental events (\cref{sec:acc}).
        The rate of uncorrelated events was used to compute the rate
        of accidental coincidences among events at any energy,
        distance, or coincidence time.
        The synthetic sample, comprised of isolated single events paired together,
        was used to estimate the fraction of accidental coincidences
        which passed the energy and distance-time (DT) cut criteria
        ($\varepsilon_{\text{total,\,acc}}$).
        The statistical uncertainty ($\sim\SI{0.1}{\percent}$) was computed
        by propagating the counting and binomial errors
        from the singles rate measurement and $\varepsilon_{\text{total,\,acc}}$.
        A systematic uncertainty (\SI{0.04}{\percent}) was estimated
        by comparing the predicted and actual number of events
        with large DT values ($\text{DT} > \SI{3}{\m}$),
        which are expected to be purely accidental coincidences.
    \item The cosmogenic isotope (\li{}/\he{}) background (\cref{subsec:li9})
        was characterized by modifying the muon veto
        to allow for events shortly following an energetic muon.
        The distribution of delays since the previous muon
        was fit using the known livetimes of \li{}, \he{}, and \boron{}
        to extract the number of events in the modified sample.
        For low- and medium-energy muons, an additional neutron event
        was required for the modified selection (``neutron tagging'').
        The final contamination of these events was computed
        by mathematically applying the nominal muon vetoes
        and the estimated efficiency of the neutron tag requirement.
        The uncertainty on the number of events is due to
        the fit uncertainty (statistical)
        and the uncertainty on the neutron tag efficiency
        (systematic).
    \item The fast-neutron background (\cref{subsec:fastn}) was characterized by
        examining events occurring shortly following muon signals
        in the outer water shield (OWS), assumed to be fast neutrons.
        The assumption was validated by comparing the OWS-tagged spectrum
        to the nominal IBD-like spectrum at energies between \SIlist{12;100}{\MeV}.
        Normalizing the OWS-tagged spectrum to the nominal spectrum
        yields the number of background events within \SIrange{1.5}{12}{\MeV}.
        Uncertainties were determined by comparing the OWS-tagged spectrum
        below \SI{12}{\MeV} to a function which was fit to the IBD-like spectrum
        above \SI{12}{\MeV} and extrapolated to the energy range of interest.
    \item The \amc{} background (\cref{subsec:amc}) was estimated by
        examining the top-bottom asymmetry in each AD.
        The spectral shape was measured during a special data run
        using a high-intensity \amc{} source.
    \item The radiogenic neutron background (\cref{subsec:radn})
        was estimated using a Monte Carlo simulation
        based on the measured masses and estimated
        radiocontamination of the PMTs and fluorocarbon paint.
        The spectrum was estimated as part of the simulation output.
\end{itemize}
