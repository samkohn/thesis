\chapter{Conclusions}
\label{ch:conclusions}

This thesis presented a measurement of
the neutrino mixing angle \thetaot{}
using observation of reactor \nuebar{} disappearance
via the IBD reaction and neutron capture on hydrogen (nH)
at the Daya Bay experiment.
After introducing the history and theory of neutrino oscillations,
the Daya Bay experimental apparatus was described.
Precise characterization of the detector response
via a thorough calibration regime
constrained the possible variations between antineutrino detectors (ADs)
to the sub-percent level for most quantities.
The event-by-event position and energy reconstruction were used
to extract physically-meaningful quantities from the raw electronics data.
Event selection cuts were introduced to select an IBD-enriched sample
based on the double coincidence of positron annihilation and nH capture,
and the irreducible backgrounds were characterized via dedicated studies.
The selected events were used as input to a model
built on the premise of near-to-far projection,
where the observations at the near halls
were propagated to the far hall ADs
to determine the predicted event count based on an assumed value of \thetaot{}.
The model was fit to the observations using a $\chi^2$ expression
based on the Poisson maximum likelihood estimator
with nuisance parameters representing systematic uncertainties
as well as the statisital fluctuations at the near hall ADs.
The best-fit value of \thetaot{} was obtained by minimizing the $\chi^2$ expression
with respect to \thetaot{} and all pull parameters,
and the uncertainty and error budget were determined
using standard statistical techniques.
The result of $\sin^22\thetaot = 0.0731^{+0.0087}_{-0.0089}$
will be discussed in \cref{sec:discussion}.
Future prospects will be presented in \cref{sec:future}.

\section{Discussion}
\label{sec:discussion}

As shown in \cref{fig:theta13_vs_t}, the nH-based measurements for $\sin^22\thetaot{}$,
including the result from this thesis, are approximately 0.01
less than the nGd-based measurements.
A variety of potential resolutions to this discrepancy have been proposed
based on mis-characterizations of detection efficiencies, backgrounds,
or systematic biases present in the nH analysis but not in the nGd analysis.

The first avenue of investiation was the accidental background,
which had a rate at the far hall ADs greater than the signal IBD rate,
and at least 2 orders of magnitude greater than the rate
of any other irreducible background (see \cref{tab:summary_event_selection}).
This background was described in \cref{sec:acc}.
The method for determining the uncorrelated event rate
was validated using simulation in \cref{subsec:sim_singles}
to within \SI{0.017}{\percent}.
Accounting for the coincidence distance distribution of the accidental background
was more difficult.
\Cref{sec:DT_cut} describes studies made using double coincidences
rejected for having a DT value larger than \SI{800}{\mm},
which were assumed to be mostly accidentals.
A minimum DT value of \SI{3}{\m} was used to ensure no residual correlated events
were selected.
Based on the uncorrelated event rate and the synthetic accidentals sample,
the predicted number of these high-DT events matched the observation
to at worst \SI{0.04}{\percent}.
Due to the true correlated events at low DT, such a correspondence
was impossible to establish at lower DT values;
however, only in pathological situations
would the DT distribution of synthetic accidentals match the true distribution
at high DT values but diverge for lower DT values.
Thus the rate of accidental events in the final sample
was validated with a systematic uncertainty of \SI{0.04}{\percent},
just half of the statistical uncertainty of \SI{0.08}{\percent}.

The second potential source of major systematic bias for the nH analysis
was the efficiency of the DT cut, defined in \cref{eq:DT}.
While the coincidence time $\Delta t$ was measured precisely
using the \SI{\sim1}{\ns} precision of the readout electronics,
the coincidence distance $\Delta r$ was much less precise,
since it depended on the reconstructed positions of the prompt and delayed event



\section{Future prospects}
\label{sec:future}
